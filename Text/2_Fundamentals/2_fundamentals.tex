\part{Fundamentals of Working in R}

This part of the course is an introduction to the different building
blocks in \R{}, objects.

\section{Objects, Modes, and Attributes \ldots \textit{Oh my!}}

The following description is typically not covered in introductions to
\R{}. It may seem weird, un-welcoming, and esoteric. In some sense, it
is. But, the reality is, if you can keep this in mind as you become
familiar with \R{} you will understand why and how things happen.

\subsection{Objects}

Within \R{}, the basic units or entities are called ``objects''. The
following are the basic objects that you will use as you learn
\R{}. There are many more.

\begin{description}

\item[vectors] These are exactly what you think they are. A sequence
  of elements of the same \textit{mode}. We'll talk about modes in a
  second. Vectors have order. You can have a vector of length 26
  (i.e.\ 26 elements) where each element is a letter in the
  alphabet. This vector's mode will be \texttt{character}.

\item[lists] Lists are similar to vectors. They are sequences of
  elements. They have order. However, each component of a list need
  not be of the same mode. You can have a list with two elements. The
  first element being a vector containing the 26 characters in the
  English alphabet and the second element being a vector contain the
  26 relative frequencies of these characters as they show up on the
  Political Science Department's website. These two vectors are not of
  the same mode and could not be combined into a single stacked
  vector. But, we can combined them in a list.

\item[matrices/arrays] Matrices are 2-dimensional rectangular
  objects. Arrays are higher dimensional ``rectangular'' objects. Each
  element must be of the same mode (e.g.\ everything is a character
  sting, logical value, numerical value, etc.)

\item[dataframes] Dataframes can be thought of as \textit{special}
  matrices. However, they technically are \textit{special} lists. In
  general, you will tend to store your empirical data in
  dataframes. These objects are two dimensional containers with the
  rows corresponding to ``observations'' and the columns corresponding
  to ``variables''.

\item[factors] These are vectors which are intended to help classify
  categorical data. They are handle differently than numerical,
  integer, or character vectors. Notice, though, factors are a
  different class of object, not a different mode.

\item[functions] Functions, although less obviously so, are also
  objects. The details of functions vary, but the general idea is
  simple. This object takes other objects (of any mode),
  performs even more basic functions on those objects and returns some
  final object (of some arbitrary mode).
\end{description}

\subsection{Modes}

\begin{description}
\item[logical] elements of either \texttt{TRUE} or \texttt{FALSE}

\item[integer] integer numbers like 1, 2, or -539

\item[numeric] real numbers; don't use 1.00, 2.00, and 3.00 when
  1, 2, and 3 will do

\item[complex] complex or imaginary numbers; you will most likely not
  use them much in applied political methodology but they can come up!

\item[character] elements made up of ``text''; each character string
  can be some arbitrary length; \texttt{``1''} is not \texttt{1} in \R{}

\item[raw] you can safely ignore this mode!

\end{description}

\subsection{Attributes}

The command \texttt{attributes()} can be used to set and extract the
attributes of an arbitrary \R{} object. Attributes can be accessed
through this general function or through particular functions which
are focused on a particular attribute. We won't cover how to do this
or why we would do this, here is an example of some attributes:
\begin{enumerate}
\item the length of a vector of any mode (\texttt{length()})
\item the size of a matrix or a higher dimensional array (\texttt{dim()})
\item the column names of a dataframe (\texttt{colnames()})
\end{enumerate}

\section{Assignment, References, Scope}
So far, we've superficially described objects in \R{}. Everything is
an object! Our data are an object. The results from a linear
regression are an object. The function that runs a linear regression
is an object.

Once we have created an object that we need, we can store it as its
own named object instead of re-computing it. So, if we want to save
the value of $2 \pi$ and recall it without having to continuously
multiple the two factors, we can create an object called
\texttt{vTwoPi} (or \texttt{BillRiker} for that matter).

After we have assigned this value to a name, we can recall it in
various ways.

\subsection{Assignment}

The standard assignment operator is \texttt{<-}. One \textit{could}
also use \texttt{=}, but you should not. We will keep \texttt{=} for
another use. Assignment can happen in the other direction,
\texttt{->}. The general form of the use of the assignment operator is
\verb= name <- object=. Although quotes aren't necessary,
\verb= "name" <- object= is identical.

An object is saved with the name \texttt{name} once this command is
run regardless of if an object of name \texttt{name} exists or
not. You can easily overwrite an object this way. You may want to or
you may not.

\subsubsection{Naming Rules}

However, when you name an object, you must follow certain rules.

Briefly, these are:

\begin{quote}
  Identifiers consist of a sequence of letters, digits, the period
  (`.') and the underscore. They must not start with a digit nor
  underscore, nor with a period followed by a
  digit.\footnote{\url{http://cran.r-project.org/doc/manuals/R-lang.html\#Identifiers}}
\end{quote}

In addition \texttt{TRUE}, \texttt{FALSE}, \texttt{Inf}, \texttt{NaN},
\texttt{NA}, \texttt{NULL}, \texttt{if}, \texttt{else}, \texttt{for},
\texttt{in}, \texttt{while}, \texttt{next}, and \texttt{break} are
reserved words.

I recommend the following convention for naming objects in \R{}. If I
want to associate the name ``foo'' or ``Foo'' with an object and it is
a vector (or factor vector), I name it \texttt{vFoo}. For objects of
the other classes, I use \texttt{lFoo}, \texttt{dfFoo}, and
\texttt{mFoo}.

The exception to this pattern is my naming of functions. Then, I use
action words and capitalization to name the function \texttt{DoFoo} or
\texttt{CalcFoo}.

\subsection{References}

If a vector object is named \texttt{vFoo} and you would like to
display or ``print'' the object, type \texttt{vFoo} and hit
\textit{enter}. Notice, \verb=print(vFoo)= will do the same
thing. \R{} automatically calls the \texttt{print()} function when you
simply enter the object name.

You can achieve the same thing with the more powerful function
\texttt{get()}.

If you are at the \R{} prompt, \R{} will help you choose the full name
of an object you are referring to. This is called tab completion. If
you type the beginning of a name of an object that \R{} knows about
and then hit \textit{tab} twice in succession, \R{} will print out a
list of object names (e.g. vectors, functions, etc.) that are
possible completions of the partial name you typed. For example, type
``\texttt{print}'' and invoke the tab completion. Look how many
objects \R{} knows about that begin with these five letters.

\subsection{Scope}

The scoping rules of the \R{} parser are slightly beyond this
tutorial. But see
\url{http://cran.r-project.org/doc/manuals/R-lang.html#Scope} to
understand them. The \R{} workspace is a hierarchy of layers. Objects
can exist in any of these layers (or environments/frames) and the
scoping rules determine which objects are being manipulated or
created. \texttt{.GlobalEnv} (an object) is the user's workspace and this is where
we will focus our attention for the time being.

\section{Asking for Help}

Asking for help in \R{} can be tricky for the following reasons:
\begin{itemize}
\item Depending on who you ask you may not like the response (e.g.\
  some people have been ridiculed on mailing out questions that are
  poorly formed). These communities have norms, if you want help,
  observe the norms before you start shooting off emails.

\item The language you use matters when you are searching. Knowing the
  right language is often half of the problem.

\item Not everyone is solving your exact problem. So you have to
  abstract away from your problem, find an applicable solution to the
  more general problem and then apply it to your specific case.
\end{itemize}

Follow these steps:
\begin{enumerate}
\item Begin looking for help/clarification in the official
  \R{} documentation.

\item If that fails, try a general web search with google. You will
  most likely want to append ``cran r'' to whatever topic you are
  looking for help with instead of just ``r''. The latter will not
  filter out unrelated sites too well.

\item Next, turn to the \R{} specific search engines.

\item Try re-wording your query. Read up on related topics to see if
  there are any hints on what might create a more successful search.

\item Ask a colleague in person. It is easier to explain these
  problems in person with code in front of you than communicating to
  someone else in the \R{} community.

\item Lastly, do your homework before you post in a public setting. It
  is not scary and people are helpful. But, they also tend not to be
  too patient with people who look for help without reading the
  mailing list or forum posting guidelines. No need to rock the boat.
\end{enumerate}

\subsection{R Documentation}
If you want to know about the object \texttt{foo}, you can type
\verb=help(``foo'')= or \verb=?foo=. If \texttt{foo} is not an object
in your current session with documentation, you will have no results.

If you want to search the \R{} help files for a concept or topic
\texttt{foo}, try \verb=??foo= or \verb=help.search(``foo'')=.

Next, try \texttt{help.start()}. This will bring up the HTML interface
to the help facilities. Here, you can find HTML versions of the \R{}
manuals which are worth searching. You can also browse
package-specific documentation through this interface.

\subsection{R Community}

\begin{description}
\item[Stack Overflow]
  <\url{http://stackoverflow.com/questions/tagged/r}> If you have
  looked for an answer to how to do something and have found nothing,
  try posting here first. Posting to the mailing list is a bad idea
  and can be embarrassing. Stack Overflow tends to be more friendly.

\item[R Meta Search] <\url{http://search.r-project.org/}> You can
  search just about every official web page and mailing list from this
  site.

\item[R Seek] <\url{http://www.rseek.org/}> Less impressive
  meta-search engine.

\item[R Graphics Gallery]
  <\url{http://addictedtor.free.fr/graphiques/}> Site with many
  examples of graphs and the code that generates them. Useful for
  getting started with fairly complex graphical output.

\item[inside-R] <\url{http://www.inside-r.org/}> Community site run by
  Revolution Analytics. Blog posts, a forum, snippets, etc.

\item[R-Forge] <\url{http://r-forge.r-project.org/}> Development
  platform for \R{} packages. You might find what you need here before
  it is found elsewhere.

\item[R-wiki] <\url{http://rwiki.sciviews.org/doku.php}> Pretty low
  traffic site site. However, many useful code snippets. Navigating
  this site can be an art.

\end{description}

\section{Exercises}
Enter all of the code in these exercises in a text file. Give it a
\texttt{.R} file extensions. Add some useful information at the top
about what this file by using comments. Run the code using RWinEdt's toolbar.

\begin{enumerate}
\item Create two numerical vectors:
\begin{verbatim}
vA <- c(1, 3, 5, 7)
vB <- (vA + 3) / 2
\end{verbatim}

\item Create a third vector by comparing the value of \texttt{vA} and \texttt{vB}:
\verb=vC <- vA > vB=

\item Use \texttt{print()} to inspect each vector. Now inspect each
  vector by referencing it on its own line.

\item What type of vector is \texttt{vC}? Try \texttt{is(vC)}.

\item What type of object is the output of \texttt{is(vC)}? Try \texttt{is(is(vC))}.

\item Coerce \texttt{vA} to a character vector with
  \texttt{as.character()}. What happens? If you get an error, use
  \texttt{?as.character}.

\item Run \verb=is.logical(is.logical(as.character(vA)))=. Does the
  output make sense? What kind of object is the output?

\item Create a matrix where \texttt{vA} and \texttt{vB} are the rows
  using \verb=rbind(vA, vB)=. Now do the same with the columns with
  \verb=cbind(vA, vB)=. Confirm that each of these is a matrix using
  either \texttt{is()} or \texttt{is.matrix()}. How do the output of
  these two differ?

\item Search for help on \texttt{data.frame()} using
  \texttt{?}. Knowing how to read \R{} help pages is a critical skill.

\item Create a dataframe from \texttt{vA}, \texttt{vB}, and \texttt{vC}.

\item Find two \R{} packages that would help you do \textit{ideal
    point estimation}.

\end{enumerate}

%%% Local Variables: 
%%% mode: latex
%%% TeX-master: "../full_course"
%%% End: 

\part{Case One: Simple Object Creation, Manipulation, Summarizing
  Objects, Plotting, and Bookkeeping}

This part of the short course is a guided walk-through some fictional
analyses. We will create some data. Recode and summarize this
data. Then, we will make some simple plots. Lastly, we learn how to
save an \R{} object for later use, and discuss object bookkeeping.

\section{Simple Object Creation and Manipulation}
\subsection{Vectors}
We will start this walk-through by creating a vector of names:

\verb=vPeople <- c("Tyson", "Jonathan1", "Gary", "Peter", "Jonathan2")=. 
\texttt{vPeople} is a vector of character strings. Create three more 
vectors as follows:
\begin{verbatim}
vEoM <- c(FALSE, TRUE, FALSE, TRUE, NA)

vAge <- c(28, 25, 25, 30, 25)

vState <- c("TX", "CT", "CA", "DE", "MO")
\end{verbatim}
Each of these vectors is of a different mode. You can verify this with
\texttt{is()}. The \texttt{c()} function lets us concantenate elements
of the same mode and form a vector.

One technical thing to note is that ``constants'' like 1 or 4 are
vectors of length 1.

Two commands you may use from time to time are \texttt{seq()} and
\texttt{rep()}. Lastly, don't forget about the use of \texttt{:} to
create consecutive integer sequences.

\subsection{Pseudo-Random Numbers and Using Functions}

To create several more ``variable'' vectors we will use \R{}'s
pseudo-RNG capabilities. Consider \texttt{?rnorm}. The most common
probability distributions will have a page like this. So, for the
Normal distribution we see that:
\begin{description}
  \item[\texttt{rnorm}] generates random values
  \item[\texttt{dnorm}] is the density function (i.e.\ $f(x)$)
  \item[\texttt{pnorm}] is the distribution function (i.e.\ $F(x)$)
  \item[\texttt{qnorm}] is the quantile function
\end{description}
Because these are so closely related they appear on the same help
page. In the case of \texttt{rnorm}, we see that the function takes
three named arguments. The names are \texttt{n}, \texttt{mean}, and
\texttt{sd}. The last two of these have default values (0 and 1,
respectively). This means that you can evaluate a call to
\texttt{rnorm} without explicitly setting a value for \texttt{mean} or
\texttt{sd}.

Run \texttt{rnorm(4)} three times at the shell prompt. Then run
\verb=a <- rnorm(4)=. Now, reference \texttt{a} at the shell prompt
four times. Why are these different?

Using the RNG capabilities we create three more vectors:
\begin{verbatim}
vVar1 <- rbinom(5, 2, .5)
vVar2 <- rnorm(length(vPeople), vAge)
vVar3 <- rnorm(sd = vAge, mean = vAge, n = length(vPeople))
\end{verbatim}
Notice that when we call the function, the arguments do not need to
follow the order they are listed in the help file exactly. How does
\R{} match arguments if they aren't always necessary and if the order
can change? The \R{} manuals explain the precise set of rules, but
there is a shortcut to not having to worry about this. If an argument
is a named argument in the help file, explicitly name it. So, run
\texttt{rnorm(n = 4)} and not \texttt{rnorm(4)}.

\subsection{Basic Mathematical Operators}
We can perform arithmatic in the natural way. The command \texttt{4 +
  6} evaluates as we would expect. Other operators like \texttt{-},
\texttt{/}, \texttt{*}, \texttt{\%\%}, \texttt{exp()}, \texttt{log()},
\texttt{\^}, and \texttt{abs()} work, too. Create this relatively
complicated object using the arithmatic operators, the objects we have
already created, and another call to the RNG.
\begin{verbatim}
vVarY <- (4 * exp(vVar1) + 
          abs(vVar2) + (vVar3) ^ (2) + 
          runif(vPeople, -1, 1)
          )
\end{verbatim}
We know that \texttt{vVar1} is a vector. What is \texttt{exp(vVar1)}?
It too is a vector. How can you check this? \R{} evaluates the
\texttt{exp} function element by element and then concantenates the
result. Similarly, \texttt{4 * exp(vVar1)} is formed by multiplying
each element in the vector we just created by 4 and then
concantenating the results. \R{} adds vectors the same way, element by
element.

\R{} has trigonometric functions like \texttt{sin()} and
\texttt{cos()}, too.

Scientific notation is achieved by typing \texttt{1e3} and
\texttt{1e-1}.

\subsection{Matrices}

Recall that matrices must be of the same mode. How are the following
two objects different? 
\begin{verbatim}
cbind(vVar1, vVar2, vVar3)
cbind(vVar1, vVar2, vVar3, vState)
\end{verbatim}

\R{} trys hard to make the code you give it work. Often times, this
means it will coerce objects into different modes so that the code can
run. Here, numerical elements are coerced to character strings.

Create the object \verb=mNums <- cbind(vVar1, vVar2, vVar3)=. We can
get information about its dimensionality with \texttt{dim()},
\texttt{nrow()}, and \texttt{ncol()}. We can refer to just the elemnts on
the main diagonal with \texttt{diag(mNums)}. Interestingly enough, we
can create a square, diagonal matrix with of size 6, for example, with
\texttt{diag(6)}. The arithmatic operators work on matrices just like
they do on vectors (i.e.\ element-wise). Try a few on \texttt{mNums}.

Now, there are also matrix-specific operations. We can transpose the
the matrix with \texttt{t()}. We perform matrix multiplication with
\texttt{\%*\%}. We can calculate the determinent of a square matrix
with \texttt{det()}. We can invert a square matrix with \texttt{solve()}

Sometimes the rows and columns have labels which makes life easier for
users. These are stored as a vector of character strings and this is,
not surprisingly, an object itself. See \texttt{?rownames} and
\texttt{?colnames}.

\subsection{Dataframes}
Create a data frame with the following code:
\begin{verbatim}
dfData <- data.frame(vPeople, vEoM, vAge, vState, 
                     vVar1, vVar2, vVar3, vVarY
                     )
\end{verbatim}
Print out the object \texttt{dfData}. Notice that character strings
become factors. The numbers remain numerical elements and are not
coerced. Lastly, we can see that the logical values have remained as
logicals. This happens because dataframes allow us to mix modes in a
way matrices do not. See \texttt{?data.frame} for more information on
creating them.

\subsection{Indexing and Subsetting}
We can refer to subsets of vector, matrix, and dataframe
objects. Vectors have a single dimension (i.e.\ length). So, we simply
specify the positions corresponding to the elements we wish to extract
from the vector. Try:

\begin{verbatim}
vPeople[1]
vPeople[-1]
vPeople[c(1,4,2)]
\end{verbatim}

Matrices have two dimensions. So, to refer to particular elements,
specify the exact row and column as in \texttt{mNums[2, 3]} or
\texttt{mNums[c(2,1), 3]}. You can refer to entire columns with
\texttt{mNums[2, ]} and entire rows with \texttt{mNums[, 3]}.

Elements in dataframes can be referred to just like they can in
matrices. But, there is some added functionality in this case. We can
still do:
\begin{verbatim}
dfData[1,1]
dfData[-1,-1]
dfData[-1,-1]
\end{verbatim}
However, the columns can be referred to in a special way. Try:
\begin{verbatim}
dfData$vPeople
dfData[, "vAge"]
\end{verbatim}
To see the names of the columns returns as character vector use
\texttt{names(dfData)}. Even though they appear to be similar,
\texttt{dfData\$vPeople} and \texttt{vPeople} are two different
objects.

We can subset dataframes based on logical tests. Typically, the
logical test we use will be dependent on the data, itself. See
\texttt{?subset}. Then, consider \verb=subset(dfData, vAge > 25)=. One
very important point to note is that the \texttt{vAge} term in the
logical test refers to the column name in \texttt{dfData} and not the
object by the same name. Confirm this by replacing the object
\texttt{vAge} with a vector that equals the old vector plus 100. Rerun
the subset command.

Moreover, we can use any arbitrary logical vector to subset any
object. Consider:
\begin{verbatim}
vPeople[c(TRUE, TRUE, FALSE, FALSE, TRUE)]
vPeople[vAge > 25]
vPeople[((1:5) * 3) < 10]
\end{verbatim}

\subsection{Logical Operators}
We just saw some logical vectors used in indexing and subsetting. In
general, they will be very useful. The common logical operators are
\texttt{<}, \texttt{>}, \texttt{|}, \texttt{\&}, \texttt{is.*()},
\texttt{!}, \texttt{<=}, \texttt{>=} and \texttt{xor()}.

Two particularly useful functions are \texttt{any()} and
\texttt{each()}. They test a vector of logicals for whether or not at
least one and every element, respectively, evaluates to \texttt{TRUE}.

\subsection{Simple Recoding}
Combining the ways to index particular elements in an object and the
set of logical operators, we are equipped to do some simple recoding
of values within objects that we have already created.

If we find out that Tyson is not from Texas, but from France (hence,
Chatagnier), we will want to change the entry for him in the state
variable. We will leave alone the state variable in the
dataframe. Instead, consider the original vector we created,
\texttt{vState}. If we want Tyson's entry to display that he is from
\texttt{FR} we can use \verb=vState[1] <- "FR"= because we know the
first entry corresponds to him. Confirm that this worked by
considering \texttt{vState}. However, this required we know his entry
was in the first row. If we didn't know this, we could use
\begin{verbatim}
vState[vPeople == "Tyson"] <- "FR"
\end{verbatim}
However, we must be careful. This will change the state to \texttt{FR}
for every Tyson there is. If we have more than one, we'd need to go
about this another way.

Alternatively, we could use the \texttt{ifelse()} function. Suppose we
have to censor the state of origin for all individuals under the age
of 26. We could use
\begin{verbatim}
vState <- ifelse(vAge < 26, NA, vState)
\end{verbatim}
To see why, consider \texttt{?ifelse}.

\section{Summary Functions}
Many times, when using \R{}, we will want to describe many numbers or
observations with some sort of summary value like the mean. We will
use different summaries for categorical, ordinal, and interval level
data. At the same time, we have functions for both univariate and
multivariate summaries.

The functions \texttt{mean()}, \texttt{median()}, \texttt{mode()},
\texttt{sd()}, \texttt{sum()}, \texttt{max()}, and \texttt{min}
describe the central tendencies and other aspects of the data. How
these functions handle missing data (i.e. \texttt{NA}) matters
greatly. Consider the following:
\begin{verbatim}
mean(vVarY)
min(vVarY)
max(vVarY)
sd(vVarY)

mean(vEoM)
mean(vEoM, na.rm = TRUE)
\end{verbatim}

However, instead of evaluating several functions for each vector we
can use \texttt{summary(dfData)} and this will give us most of the
information we might have wanted. \R{} wisely customize the summary
for each variable based on the kind of data it is. For factors and
logicals, we find out about frequencies. For numerical data, we get a
description of the distribution.

Beyond these univariate measures, we can use \texttt{cor()} to
calculate the correlation coefficient between two numerical vectors or
\texttt{table()} to count the number of observations occurring within
each unique classification between multiple factor vectors. Consider
the following:
\begin{verbatim}
cor(vVarY, vVar1)
table(dfData$vAge, dfData$vState)
\end{verbatim}

If we are interested in conditional summaries, we need to do a little
more work. There are two natural ways to approach questions like, what
is the average value of \texttt{vVarY} within people of the same
age. We know there are three different groups because of
\texttt{unique(dfData\$vAge)}. What does \texttt{unique()} do?

We could subset the dataframe according to age, then run
\texttt{summary} or \texttt{mean}. Or, instead, we could use a very
convenient function called \texttt{tapply()}. Consider
\texttt{?tapply} and then the code below.
\begin{verbatim}
mean(subset(dfData, vAge == 25)$vVarY)
tapply(dfData$vVarY, dfData$vAge, mean)
\end{verbatim}

\section{Basic Plots}
Although the numerical/textual summaries of our data will be helpful,
we often look to make figures to communicate some aspect of our date
or some result to others. For now, we will simply go through different
kinds of plots without concerning ourselves with customization.

If we have univariate data, we could use a boxplot or a histogram. Consider
the following:
\begin{verbatim}
boxplot(rnorm(n = 50))
hist(rnorm(n = 50))
\end{verbatim}

If we were so inclined, we could simply plot the values of interest
against an index. Try:
\begin{verbatim}
plot(rnorm(n = 50))
\end{verbatim}

If we have bivariate data, we might want a scatterplot with lines
connecting observations in sequence. Consider:
\begin{verbatim}
plot(rnorm(n = 50), rnorm(n = 50), type = "b")
\end{verbatim}

For three dimensional data, we can generate a surface plot. Use:
\begin{verbatim}
persp(z = matrix(data = sort(rnorm(100, sd = 3)), ncol = 10))
\end{verbatim}

The options are many and
\url{http://cran.r-project.org/web/views/Graphics.html} is a good link
to read. If you know the kind of plot you would like to create, it
will most likely either exist already or you can create it. We will
cover building plots from the ground up and customization later.

\section{Object Bookkeeping}

Recall, just about everything in \R{} is an object. And, by now, we've
created a number of objects. To display all of the objects type
\texttt{ls()}. What is the class of the output of this function call?
If you would like to remove an object named \texttt{foo}, try
\texttt{rm(foo)}. If you would like to remove every object, use
\texttt{rm(list = ls())}. Some folks use this at the top of each
script to make sure their \R{} session is fresh. I would recommend
this.

Now, let's create an object that will contain all of our work. A list
is the best choice here. First, we initialize a list. Second, we will
save the dataframe as one element in this list. Third, and last we
will leave a note for ourselves as a character string.
\begin{verbatim}
lOutput <- list()
lOutput$data <- dfData
lOutput$comment <- "today's work was very rewarding"
lOutput
\end{verbatim}
Now, we can reference the elements in the list by their names and the
\texttt{\$} operator.

To save the object \texttt{lOutput}, first look at
\texttt{?save}. Then, run:
\begin{verbatim}
save(lOutput, file = "Z:/mydata.Rdata"}
\end{verbatim}

%%% Local Variables: 
%%% mode: latex
%%% TeX-master: "../full_course"
%%% End: 

%%%%%%%%%%%%%%%%%%%%%%%%%%%%%%%%%%%%%%%%%%%%%%%%%%%%%%%%%%%%%%%%%%%%%%%%%%%%%%%% 
%%%%%%%%%%%%%%%%%%%%%%%%%%%%%%%%%%% PREAMBLE %%%%%%%%%%%%%%%%%%%%%%%%%%%%%%%%%%%
%%%%%%%%%%%%%%%%%%%%%%%%%%%%%%%%%%%%%%%%%%%%%%%%%%%%%%%%%%%%%%%%%%%%%%%%%%%%%%%% 


%%%%%%%%%%%%%%%%%%%%%%%%%%%%%%%%%%%%%%%%% 
%%%%%%%%%%%%%%%%% CLASS %%%%%%%%%%%%%%%%%
%%%%%%%%%%%%%%%%%%%%%%%%%%%%%%%%%%%%%%%%% 

\documentclass[10pt,letterpaper]{article}

%%%%%%%%%%%%%%%%%%%%%%%%%%%%%%%%%%%%%%%% 
%%%%%%%%%%%%%%% PACKAGES %%%%%%%%%%%%%%%
%%%%%%%%%%%%%%%%%%%%%%%%%%%%%%%%%%%%%%%% 

\usepackage[debugshow, final]{graphicx}
\usepackage{amsmath, amssymb, amsthm, amsfonts} 
\usepackage{enumerate}
\usepackage{fancybox} 
\usepackage{ltablex}
\usepackage{fancyhdr}
\usepackage{setspace} 
% \usepackage[top=1in,left=1in,bottom=1in,right=1in]{geometry} 
\usepackage{natbib}
\usepackage{rotating}
\usepackage{setspace}
\usepackage{multicol}
\usepackage{textcomp} 
\usepackage{pdflscape}
\usepackage{lscape}
\usepackage[table]{xcolor}
\usepackage[T1]{fontenc}
\usepackage{subfigure}
\usepackage{hyperref}
\usepackage{cclicenses}
\usepackage[all]{hypcap}
\usepackage{listings}
% \usepackage{pxfonts}

%%%%%%%%%%%%%%%%%%%%%%%%%%%%%%%%%%%%%%%% 
%%%%%%%%%%%%%%% SETTINGS %%%%%%%%%%%%%%%
%%%%%%%%%%%%%%%%%%%%%%%%%%%%%%%%%%%%%%%% 

\pagestyle{fancy}
% \lhead{}
% \chead{}
% \rhead{}
% \cfoot{\thepage}

\author{Prepared for the starlab by J.P.\ Olmsted}

\title{\tsl{}'s Introduction to \R: A Short Course} 

\date{Updated: \today}

%%%%%%%%%%%%%%%%%%%%%%%%%%%%%%%%%%%%%%%%% 
%%%%%%%%%%%%% CUSTOM MACROS %%%%%%%%%%%%%
%%%%%%%%%%%%%%%%%%%%%%%%%%%%%%%%%%%%%%%%% 

\let\margin\marginpar
\newcommand\myMargin[1]{
  \margin{\singlespace\raggedright\scriptsize
    \textsf{#1}}
} 
\renewcommand{\marginpar}[1]{\myMargin{#1}}

\newcommand\blurb[1]{
  \begin{center}
    \parbox[h]{.75\linewidth}{
      \footnotesize \singlespace
      #1  
    }
  \end{center}
}

\newcommand\R[0]{\texttt{R}}

\newcommand\tsl[0]{\textbf{the star lab}}

%%%%%%%%%%%%%%%%%%%%%%%%%%%%%%%%%%%%%%%%%%%%%%%%%%%%%%%%%%%%%%%%%%%%%%%%%%%%%%% 
%%%%%%%%%%%%%%%%%%%%%%%%%%%%%%%%%%% CONTENT %%%%%%%%%%%%%%%%%%%%%%%%%%%%%%%%%%%
%%%%%%%%%%%%%%%%%%%%%%%%%%%%%%%%%%%%%%%%%%%%%%%%%%%%%%%%%%%%%%%%%%%%%%%%%%%%%%% 

\begin{document}

\begin{titlepage}
\maketitle 
\tableofcontents
\end{titlepage}

\section*{Introduction to this Short Course}

\paragraph{About.}This document was prepared by the author during his
tenure as \tsl{} Fellow for the incoming graduate student cohort in
the Political Science department at the University of
Rochester. Previous versions of the short course and the corresponding
materials are available at \tsl{}'s website
(\url{http://www.rochester.edu/college/psc/thestarlab/main/index.php}). This
document is an adaptation of Arthur Spirling's version. In addition to
the supplemental files used for the various exercises included in this
version, this document is available from both the starlab's website
and the author's website
(\url{http://www.rochester.edu/college/gradstudents/jolmsted/}).

\paragraph{Rights.}This document is released under the Creative
Commons Attribution license \by.

\paragraph{Comments.}If you have any comments, questions, or concerns
regarding this document please contact Jonathan Olmsted
(\url{jpolmsted@gmail.com}).

\paragraph{Description of Course.} This short course was covered in
five days. Parts 1 and 2 are covered in one day and the remaining
parts are covered on their own days. Part 1 discusses the whats and
whys of \R{}. Part 2 is a brief introduction to the fact that
everything in \R{} is an object and that objects have traits like
\textit{classes} and \textit{modes}. Part 3 is the first journey into
practice with the \R{} language. We walk through an example of how to
create and manipulate various objects in a way related to 

\pagebreak

\part{An Introduction to Using \R}

This part of the short course introduces you to \R{} and provides a
basic overview of its use.

\section{What is \R?}

\begin{itemize}
\item a statistical scripting language (i.e.\ it was created with data
  in mind and you can use the language interactively)

\item a statistical environment (i.e.\ it is a program you open up and
  interact with)

\item open-source, free, compatible with most platforms, and
  constantly updated

\item the most common platform for quantitative work in political
  science (or at least I am claiming this)

\end{itemize}

You can find out more about \R{} at \url{http://www.r-project.org/}.

\section{Why Would You Use \R{}?}
\begin{itemize}
\item you will have to use it for PSC 405 and PSC 505 problem sets

\item it is the ``go to'' program for non-standard numerical analysis
  in political science

\item it was created with data in mind (so you aren't forcing a square
  peg in a round hole)

\item the language and your code is as much a way of communicating a
  set of instructions to other researchers as it is to your computer

\item the graphics capabilities of \R{} are un-rivaled among statistics
  software

\item \R{} can be anything you want it to be because it is so easily
  extended

\end{itemize}

Since you are now convinced that you would like to use \R{}, you can
download it at \url{http://www.r-project.org/index.html}. However, you
won't need to download it if you are in \tsl{}. It is
already installed on these computers.

\section{\R~and Your Workflow}

The role that \R{} will play in your work and how it will fit into your
workflow will vary. However, there is a general structure to it all.

\begin{enumerate}
\item come up with a numerical problem to solve
\item conceptualize a set of steps to solve this problem
\item translate these steps into the \R{} scripting language
\item run your \R{} code
\item debug your \R{} code
\item produce graphics (if applicable)
\item consume you results (either textual or graphical)
\end{enumerate}

\paragraph{Step 1.} \R{} will not tell you what you want to do. It can
help you get there, but you simply must know where you want to go. The
problem to be solved can range from \textit{calculating the sampling
  distribution of the sample mean of samples of five random draws from
  a distribution} to \textit{identifying the treatment effect of being exposed
to negative campaign advertisements on voter turnout}. In general, it
is a bad sign to have already opened \R{} if you don't know what you
want to do.

\paragraph{Step 2.} If \R{} is your first computer language, this step
will likely be tricky. This is all about translating a
human understanding of a problem into a machine-ready algorithm. For
some people this is easier than it is for others. For all people,
though, practice makes it easier.

\paragraph{Step 3.} This will be the hardest step. As you use \R{}, you
understanding of what is and is not possible in \R{} will evolve. The
code you write will take on incredibly different styles.

\par Because you are writing your \R{} code in a text-editor it will
be a plain text file (like in \LaTeX{}!).

\par When you write your \R{} code, I recommend doing several things:
\begin{itemize}
\item \textbf{do not use} the feature-less script editor in \R{}
  itself

\item \textbf{do use} a feature-full editor that integrates well
  enough with \R{} (e.g.\ RWinEDT on \tsl{} computers)

\item \textbf{do use} an editor to write your code and then run it at
  the \R{} prompt (this maintains transparency and reproducibility).

\end{itemize}

\paragraph{Step 4.} Running your code can happen at an interactive
\R{} shell or in ``batch mode''. The latter option will most likely
not be used by you any time soon. For our purposes we will run code
interactively. But, that is different from performing operations ``on
the fly''. If you write your code in RWinEdt, the toolbar icons
provide a simple way of running your already written code at the
interactive shell.

\paragraph{Step 5.} There is a built-in debugger in R and there are
some external packages which provide debugging facilities. I neither
use nor recommend these. For the most part, there is nothing you can't
debug with just some cleverness and the interactive command line. In
reality, Steps 4 and 5 are iterative and repetitive. In it's idealized
form, you might write your code in a single shot and run it without
errors. In practice, this will not happen. The point is, though, that
for any given project, you'll have it set up so that it can finally
run without intervention or correcting. It will be as if you wrote
completely correct code and sent it all to \R{}.

\paragraph{Step 6.} \R{} is known for the ability to produce great
graphical output. There are several different ``frameworks'' to
produce graphics. We will only cover the \textit{base} graphics in
this tutorial.

\paragraph{Step 7.} How you consume your output is up to you. But,
consumption is not a one-shot deal. Therefore, part of this step is
ensuring reproducibility. Also, you will want to ``export'' the
product of your hard work to a non-\R{} format. This can be a plain
text file of your results or a \texttt{.pdf} file with a figure you
created.

\section{The VERY Basics of the \R{} Interpreter}

\subsection{The Prompts}
\R{} tells you it is ready for input by displaying a
``\texttt{>}''. \R{} knows its own language, so if you hit
\textit{enter} and you expect something to evaluate, but \R{} keeps
giving you a ``\texttt{+}'' (to tell you it needs more) there is a
syntax error. To get out of the perpetual ``\texttt{+}'' prompts, hit
\textit{ctrl} + \textit{c} or \textit{esc}. When you get this problem,
there is likely to be a missing ``\verb="='', ``\verb=,='', or
``\verb=)=''.

At the end of a command, you hit \textit{enter} if you are at the
command line. If you are writing your script file in a text editor,
simply hit \textit{enter} to add a line break and start a new line.

\subsection{Comments}
\R{} interprets anything after a ``\texttt{\#}'' as a comment. The
comment ends and \R{} begins to interpret code once a new line is
started. Multi-line comments require a new comment character on each
line.

\subsection{Spaces}
\R{}, in general, ignores spaces between correctly entered objects and
operators. Spaces are not part of the names of objects or
operators. Spaces may be contained in character strings and they
matter.

\subsection{Multiple Commands on a Physical Line}
In \R{}, you can place two (or more) complete sets of commands on the
same physical line (i.e.\ without hitting \textit{enter}). You do this
by separating them with a ``\verb=;=''. However, this is not
recommended and the utility of it is questionable.


\section{Exercises}

These short exercises are intended to familiarize you with entering
commands at the prompt and the various kinds of windows that might
open as a result.

\begin{enumerate}
\item Start \R{} on the machines in \tsl{}.

\item If the \texttt{RWinEdt} package is not automatically loaded,
  type \texttt{library(RWinEdt)}. Hit \textit{enter}.

\item Type \texttt{sessionInfo()} at the command prompt (i.e.\
  \texttt{>}) and hit \textit{enter}. Your are running the function
  \texttt{sessionInfo} without any arguments. \R{} is ``computing''
  the output of the function and then ``printing'' the output with an
  ``appropriate'' method.

\item Type \texttt{citation()} at the command prompt and hit
  \textit{enter}. Recall the mention of BibTeX from the \LaTeX{}
  course? \R{} is ``computing'' the output of the function
  \texttt{citation} and then ``printing'' the output with an
  ``appropriate'' method.

\item Enter \texttt{demo(graphics)} at the prompt. Progress through
  the various demonstrated plots until you have seen them all. Close
  the plot window. \R{} is ``computing'' the output of the function
  \texttt{demo} which has been passed the argument
  \texttt{graphics}. In fact, \texttt{graphics} is a package which
  contains functions for creating graphics. Would you like to know
  something about it? Enter \texttt{?graphics} at the command
  line. This displays the help page for the \texttt{graphics}
  package. Close this.

\item Enter the following in WinEdt. You do not need to save this R
  script.

\begin{verbatim}
###
### Region 1
###
# is this 2 or 3?
1 + 1 # + 1
###

###
### Region 2
###
  1+ 1   +
1
###

###
### Region 3
###
"asdf " == "asdf"
###
\end{verbatim}

\item Evaluate each ``region'' one at a time. So, select the region
  with your cursor, and press the RWinEdt toolbar button with the word
  ``paste'' on it. This pastes the selected region into the \R{}
  shell.

\item Type \texttt{1 +} and hit \textit{enter}. Hit \textit{enter} a
  number of times. How do you get out of this?

\item Enter \texttt{q()} at the prompt. Respond ``\texttt{N}'' (no).
\end{enumerate}

%%% Local Variables: 
%%% mode: latex
%%% TeX-master: "../full_course"
%%% End: 
 

\clearpage
\pagebreak

\part{Fundamentals of Working in R}

This part of the course is an introduction to the different building
blocks in \R{}, objects.

\section{Objects, Modes, and Attributes \ldots \textit{Oh my!}}

The following description is typically not covered in introductions to
\R{}. It may seem weird, un-welcoming, and esoteric. In some sense, it
is. But, the reality is, if you can keep this in mind as you become
familiar with \R{} you will understand why and how things happen.

\subsection{Objects}

Within \R{}, the basic units or entities are called ``objects''. The
following are the basic objects that you will use as you learn
\R{}. There are many more.

\begin{description}

\item[vectors] These are exactly what you think they are. A sequence
  of elements of the same \textit{mode}. We'll talk about modes in a
  second. Vectors have order. You can have a vector of length 26
  (i.e.\ 26 elements) where each element is a letter in the
  alphabet. This vector's mode will be \texttt{character}.

\item[lists] Lists are similar to vectors. They are sequences of
  elements. They have order. However, each component of a list need
  not be of the same mode. You can have a list with two elements. The
  first element being a vector containing the 26 characters in the
  English alphabet and the second element being a vector contain the
  26 relative frequencies of these characters as they show up on the
  Political Science Department's website. These two vectors are not of
  the same mode and could not be combined into a single stacked
  vector. But, we can combined them in a list.

\item[matrices/arrays] Matrices are 2-dimensional rectangular
  objects. Arrays are higher dimensional ``rectangular'' objects. Each
  element must be of the same mode (e.g.\ everything is a character
  sting, logical value, numerical value, etc.)

\item[dataframes] Dataframes can be thought of as \textit{special}
  matrices. However, they technically are \textit{special} lists. In
  general, you will tend to store your empirical data in
  dataframes. These objects are two dimensional containers with the
  rows corresponding to ``observations'' and the columns corresponding
  to ``variables''.

\item[factors] These are vectors which are intended to help classify
  categorical data. They are handle differently than numerical,
  integer, or character vectors. Notice, though, factors are a
  different class of object, not a different mode.

\item[functions] Functions, although less obviously so, are also
  objects. The details of functions vary, but the general idea is
  simple. This object takes other objects (of any mode),
  performs even more basic functions on those objects and returns some
  final object (of some arbitrary mode).
\end{description}

\subsection{Modes}

\begin{description}
\item[logical] elements of either \texttt{TRUE} or \texttt{FALSE}

\item[integer] integer numbers like 1, 2, or -539

\item[numeric] real numbers; don't use 1.00, 2.00, and 3.00 when
  1, 2, and 3 will do

\item[complex] complex or imaginary numbers; you will most likely not
  use them much in applied political methodology but they can come up!

\item[character] elements made up of ``text''; each character string
  can be some arbitrary length; \texttt{``1''} is not \texttt{1} in \R{}

\item[raw] you can safely ignore this mode!

\end{description}

\subsection{Attributes}

The command \texttt{attributes()} can be used to set and extract the
attributes of an arbitrary \R{} object. Attributes can be accessed
through this general function or through particular functions which
are focused on a particular attribute. We won't cover how to do this
or why we would do this, here is an example of some attributes:
\begin{enumerate}
\item the length of a vector of any mode (\texttt{length()})
\item the size of a matrix or a higher dimensional array (\texttt{dim()})
\item the column names of a dataframe (\texttt{colnames()})
\end{enumerate}

\section{Assignment, References, Scope}
So far, we've superficially described objects in \R{}. Everything is
an object! Our data are an object. The results from a linear
regression are an object. The function that runs a linear regression
is an object.

Once we have created an object that we need, we can store it as its
own named object instead of re-computing it. So, if we want to save
the value of $2 \pi$ and recall it without having to continuously
multiple the two factors, we can create an object called
\texttt{vTwoPi} (or \texttt{BillRiker} for that matter).

After we have assigned this value to a name, we can recall it in
various ways.

\subsection{Assignment}

The standard assignment operator is \texttt{<-}. One \textit{could}
also use \texttt{=}, but you should not. We will keep \texttt{=} for
another use. Assignment can happen in the other direction,
\texttt{->}. The general form of the use of the assignment operator is
\verb= name <- object=. Although quotes aren't necessary,
\verb= "name" <- object= is identical.

An object is saved with the name \texttt{name} once this command is
run regardless of if an object of name \texttt{name} exists or
not. You can easily overwrite an object this way. You may want to or
you may not.

\subsubsection{Naming Rules}

However, when you name an object, you must follow certain rules.

Briefly, these are:

\begin{quote}
  Identifiers consist of a sequence of letters, digits, the period
  (`.') and the underscore. They must not start with a digit nor
  underscore, nor with a period followed by a
  digit.\footnote{\url{http://cran.r-project.org/doc/manuals/R-lang.html\#Identifiers}}
\end{quote}

In addition \texttt{TRUE}, \texttt{FALSE}, \texttt{Inf}, \texttt{NaN},
\texttt{NA}, \texttt{NULL}, \texttt{if}, \texttt{else}, \texttt{for},
\texttt{in}, \texttt{while}, \texttt{next}, and \texttt{break} are
reserved words.

I recommend the following convention for naming objects in \R{}. If I
want to associate the name ``foo'' or ``Foo'' with an object and it is
a vector (or factor vector), I name it \texttt{vFoo}. For objects of
the other classes, I use \texttt{lFoo}, \texttt{dfFoo}, and
\texttt{mFoo}.

The exception to this pattern is my naming of functions. Then, I use
action words and capitalization to name the function \texttt{DoFoo} or
\texttt{CalcFoo}.

\subsection{References}

If a vector object is named \texttt{vFoo} and you would like to
display or ``print'' the object, type \texttt{vFoo} and hit
\textit{enter}. Notice, \verb=print(vFoo)= will do the same
thing. \R{} automatically calls the \texttt{print()} function when you
simply enter the object name.

You can achieve the same thing with the more powerful function
\texttt{get()}.

If you are at the \R{} prompt, \R{} will help you choose the full name
of an object you are referring to. This is called tab completion. If
you type the beginning of a name of an object that \R{} knows about
and then hit \textit{tab} twice in succession, \R{} will print out a
list of object names (e.g. vectors, functions, etc.) that are
possible completions of the partial name you typed. For example, type
``\texttt{print}'' and invoke the tab completion. Look how many
objects \R{} knows about that begin with these five letters.

\subsection{Scope}

The scoping rules of the \R{} parser are slightly beyond this
tutorial. But see
\url{http://cran.r-project.org/doc/manuals/R-lang.html#Scope} to
understand them. The \R{} workspace is a hierarchy of layers. Objects
can exist in any of these layers (or environments/frames) and the
scoping rules determine which objects are being manipulated or
created. \texttt{.GlobalEnv} (an object) is the user's workspace and this is where
we will focus our attention for the time being.

\section{Asking for Help}

Asking for help in \R{} can be tricky for the following reasons:
\begin{itemize}
\item Depending on who you ask you may not like the response (e.g.\
  some people have been ridiculed on mailing out questions that are
  poorly formed). These communities have norms, if you want help,
  observe the norms before you start shooting off emails.

\item The language you use matters when you are searching. Knowing the
  right language is often half of the problem.

\item Not everyone is solving your exact problem. So you have to
  abstract away from your problem, find an applicable solution to the
  more general problem and then apply it to your specific case.
\end{itemize}

Follow these steps:
\begin{enumerate}
\item Begin looking for help/clarification in the official
  \R{} documentation.

\item If that fails, try a general web search with google. You will
  most likely want to append ``cran r'' to whatever topic you are
  looking for help with instead of just ``r''. The latter will not
  filter out unrelated sites too well.

\item Next, turn to the \R{} specific search engines.

\item Try re-wording your query. Read up on related topics to see if
  there are any hints on what might create a more successful search.

\item Ask a colleague in person. It is easier to explain these
  problems in person with code in front of you than communicating to
  someone else in the \R{} community.

\item Lastly, do your homework before you post in a public setting. It
  is not scary and people are helpful. But, they also tend not to be
  too patient with people who look for help without reading the
  mailing list or forum posting guidelines. No need to rock the boat.
\end{enumerate}

\subsection{R Documentation}
If you want to know about the object \texttt{foo}, you can type
\verb=help(``foo'')= or \verb=?foo=. If \texttt{foo} is not an object
in your current session with documentation, you will have no results.

If you want to search the \R{} help files for a concept or topic
\texttt{foo}, try \verb=??foo= or \verb=help.search(``foo'')=.

Next, try \texttt{help.start()}. This will bring up the HTML interface
to the help facilities. Here, you can find HTML versions of the \R{}
manuals which are worth searching. You can also browse
package-specific documentation through this interface.

\subsection{R Community}

\begin{description}
\item[Stack Overflow]
  <\url{http://stackoverflow.com/questions/tagged/r}> If you have
  looked for an answer to how to do something and have found nothing,
  try posting here first. Posting to the mailing list is a bad idea
  and can be embarrassing. Stack Overflow tends to be more friendly.

\item[R Meta Search] <\url{http://search.r-project.org/}> You can
  search just about every official web page and mailing list from this
  site.

\item[R Seek] <\url{http://www.rseek.org/}> Less impressive
  meta-search engine.

\item[R Graphics Gallery]
  <\url{http://addictedtor.free.fr/graphiques/}> Site with many
  examples of graphs and the code that generates them. Useful for
  getting started with fairly complex graphical output.

\item[inside-R] <\url{http://www.inside-r.org/}> Community site run by
  Revolution Analytics. Blog posts, a forum, snippets, etc.

\item[R-Forge] <\url{http://r-forge.r-project.org/}> Development
  platform for \R{} packages. You might find what you need here before
  it is found elsewhere.

\item[R-wiki] <\url{http://rwiki.sciviews.org/doku.php}> Pretty low
  traffic site site. However, many useful code snippets. Navigating
  this site can be an art.

\end{description}

\section{Exercises}
Enter all of the code in these exercises in a text file. Give it a
\texttt{.R} file extensions. Add some useful information at the top
about what this file by using comments. Run the code using RWinEdt's toolbar.

\begin{enumerate}
\item Create two numerical vectors:
\begin{verbatim}
vA <- c(1, 3, 5, 7)
vB <- (vA + 3) / 2
\end{verbatim}

\item Create a third vector by comparing the value of \texttt{vA} and \texttt{vB}:
\verb=vC <- vA > vB=

\item Use \texttt{print()} to inspect each vector. Now inspect each
  vector by referencing it on its own line.

\item What type of vector is \texttt{vC}? Try \texttt{is(vC)}.

\item What type of object is the output of \texttt{is(vC)}? Try \texttt{is(is(vC))}.

\item Coerce \texttt{vA} to a character vector with
  \texttt{as.character()}. What happens? If you get an error, use
  \texttt{?as.character}.

\item Run \verb=is.logical(is.logical(as.character(vA)))=. Does the
  output make sense? What kind of object is the output?

\item Create a matrix where \texttt{vA} and \texttt{vB} are the rows
  using \verb=rbind(vA, vB)=. Now do the same with the columns with
  \verb=cbind(vA, vB)=. Confirm that each of these is a matrix using
  either \texttt{is()} or \texttt{is.matrix()}. How do the output of
  these two differ?

\item Search for help on \texttt{data.frame()} using
  \texttt{?}. Knowing how to read \R{} help pages is a critical skill.

\item Create a dataframe from \texttt{vA}, \texttt{vB}, and \texttt{vC}.

\item Find two \R{} packages that would help you do \textit{ideal
    point estimation}.

\end{enumerate}

%%% Local Variables: 
%%% mode: latex
%%% TeX-master: "../full_course"
%%% End: 


\clearpage
\pagebreak

\part{Case One: Simple Object Creation, Manipulation, Summarizing
  Objects, Plotting, and Bookkeeping}

This part of the short course is a guided walk-through some fictional
analyses. We will create some data. Recode and summarize this
data. Then, we will make some simple plots. Lastly, we learn how to
save an \R{} object for later use, and discuss object bookkeeping.

\section{Simple Object Creation and Manipulation}
\subsection{Vectors}
We will start this walk-through by creating a vector of names:

\verb=vPeople <- c("Tyson", "Jonathan1", "Gary", "Peter", "Jonathan2")=. 
\texttt{vPeople} is a vector of character strings. Create three more 
vectors as follows:
\begin{verbatim}
vEoM <- c(FALSE, TRUE, FALSE, TRUE, NA)

vAge <- c(28, 25, 25, 30, 25)

vState <- c("TX", "CT", "CA", "DE", "MO")
\end{verbatim}
Each of these vectors is of a different mode. You can verify this with
\texttt{is()}. The \texttt{c()} function lets us concantenate elements
of the same mode and form a vector.

One technical thing to note is that ``constants'' like 1 or 4 are
vectors of length 1.

Two commands you may use from time to time are \texttt{seq()} and
\texttt{rep()}. Lastly, don't forget about the use of \texttt{:} to
create consecutive integer sequences.

\subsection{Pseudo-Random Numbers and Using Functions}

To create several more ``variable'' vectors we will use \R{}'s
pseudo-RNG capabilities. Consider \texttt{?rnorm}. The most common
probability distributions will have a page like this. So, for the
Normal distribution we see that:
\begin{description}
  \item[\texttt{rnorm}] generates random values
  \item[\texttt{dnorm}] is the density function (i.e.\ $f(x)$)
  \item[\texttt{pnorm}] is the distribution function (i.e.\ $F(x)$)
  \item[\texttt{qnorm}] is the quantile function
\end{description}
Because these are so closely related they appear on the same help
page. In the case of \texttt{rnorm}, we see that the function takes
three named arguments. The names are \texttt{n}, \texttt{mean}, and
\texttt{sd}. The last two of these have default values (0 and 1,
respectively). This means that you can evaluate a call to
\texttt{rnorm} without explicitly setting a value for \texttt{mean} or
\texttt{sd}.

Run \texttt{rnorm(4)} three times at the shell prompt. Then run
\verb=a <- rnorm(4)=. Now, reference \texttt{a} at the shell prompt
four times. Why are these different?

Using the RNG capabilities we create three more vectors:
\begin{verbatim}
vVar1 <- rbinom(5, 2, .5)
vVar2 <- rnorm(length(vPeople), vAge)
vVar3 <- rnorm(sd = vAge, mean = vAge, n = length(vPeople))
\end{verbatim}
Notice that when we call the function, the arguments do not need to
follow the order they are listed in the help file exactly. How does
\R{} match arguments if they aren't always necessary and if the order
can change? The \R{} manuals explain the precise set of rules, but
there is a shortcut to not having to worry about this. If an argument
is a named argument in the help file, explicitly name it. So, run
\texttt{rnorm(n = 4)} and not \texttt{rnorm(4)}.

\subsection{Basic Mathematical Operators}
We can perform arithmatic in the natural way. The command \texttt{4 +
  6} evaluates as we would expect. Other operators like \texttt{-},
\texttt{/}, \texttt{*}, \texttt{\%\%}, \texttt{exp()}, \texttt{log()},
\texttt{\^}, and \texttt{abs()} work, too. Create this relatively
complicated object using the arithmatic operators, the objects we have
already created, and another call to the RNG.
\begin{verbatim}
vVarY <- (4 * exp(vVar1) + 
          abs(vVar2) + (vVar3) ^ (2) + 
          runif(vPeople, -1, 1)
          )
\end{verbatim}
We know that \texttt{vVar1} is a vector. What is \texttt{exp(vVar1)}?
It too is a vector. How can you check this? \R{} evaluates the
\texttt{exp} function element by element and then concantenates the
result. Similarly, \texttt{4 * exp(vVar1)} is formed by multiplying
each element in the vector we just created by 4 and then
concantenating the results. \R{} adds vectors the same way, element by
element.

\R{} has trigonometric functions like \texttt{sin()} and
\texttt{cos()}, too.

Scientific notation is achieved by typing \texttt{1e3} and
\texttt{1e-1}.

\subsection{Matrices}

Recall that matrices must be of the same mode. How are the following
two objects different? 
\begin{verbatim}
cbind(vVar1, vVar2, vVar3)
cbind(vVar1, vVar2, vVar3, vState)
\end{verbatim}

\R{} trys hard to make the code you give it work. Often times, this
means it will coerce objects into different modes so that the code can
run. Here, numerical elements are coerced to character strings.

Create the object \verb=mNums <- cbind(vVar1, vVar2, vVar3)=. We can
get information about its dimensionality with \texttt{dim()},
\texttt{nrow()}, and \texttt{ncol()}. We can refer to just the elemnts on
the main diagonal with \texttt{diag(mNums)}. Interestingly enough, we
can create a square, diagonal matrix with of size 6, for example, with
\texttt{diag(6)}. The arithmatic operators work on matrices just like
they do on vectors (i.e.\ element-wise). Try a few on \texttt{mNums}.

Now, there are also matrix-specific operations. We can transpose the
the matrix with \texttt{t()}. We perform matrix multiplication with
\texttt{\%*\%}. We can calculate the determinent of a square matrix
with \texttt{det()}. We can invert a square matrix with \texttt{solve()}

Sometimes the rows and columns have labels which makes life easier for
users. These are stored as a vector of character strings and this is,
not surprisingly, an object itself. See \texttt{?rownames} and
\texttt{?colnames}.

\subsection{Dataframes}
Create a data frame with the following code:
\begin{verbatim}
dfData <- data.frame(vPeople, vEoM, vAge, vState, 
                     vVar1, vVar2, vVar3, vVarY
                     )
\end{verbatim}
Print out the object \texttt{dfData}. Notice that character strings
become factors. The numbers remain numerical elements and are not
coerced. Lastly, we can see that the logical values have remained as
logicals. This happens because dataframes allow us to mix modes in a
way matrices do not. See \texttt{?data.frame} for more information on
creating them.

\subsection{Indexing and Subsetting}
We can refer to subsets of vector, matrix, and dataframe
objects. Vectors have a single dimension (i.e.\ length). So, we simply
specify the positions corresponding to the elements we wish to extract
from the vector. Try:

\begin{verbatim}
vPeople[1]
vPeople[-1]
vPeople[c(1,4,2)]
\end{verbatim}

Matrices have two dimensions. So, to refer to particular elements,
specify the exact row and column as in \texttt{mNums[2, 3]} or
\texttt{mNums[c(2,1), 3]}. You can refer to entire columns with
\texttt{mNums[2, ]} and entire rows with \texttt{mNums[, 3]}.

Elements in dataframes can be referred to just like they can in
matrices. But, there is some added functionality in this case. We can
still do:
\begin{verbatim}
dfData[1,1]
dfData[-1,-1]
dfData[-1,-1]
\end{verbatim}
However, the columns can be referred to in a special way. Try:
\begin{verbatim}
dfData$vPeople
dfData[, "vAge"]
\end{verbatim}
To see the names of the columns returns as character vector use
\texttt{names(dfData)}. Even though they appear to be similar,
\texttt{dfData\$vPeople} and \texttt{vPeople} are two different
objects.

We can subset dataframes based on logical tests. Typically, the
logical test we use will be dependent on the data, itself. See
\texttt{?subset}. Then, consider \verb=subset(dfData, vAge > 25)=. One
very important point to note is that the \texttt{vAge} term in the
logical test refers to the column name in \texttt{dfData} and not the
object by the same name. Confirm this by replacing the object
\texttt{vAge} with a vector that equals the old vector plus 100. Rerun
the subset command.

Moreover, we can use any arbitrary logical vector to subset any
object. Consider:
\begin{verbatim}
vPeople[c(TRUE, TRUE, FALSE, FALSE, TRUE)]
vPeople[vAge > 25]
vPeople[((1:5) * 3) < 10]
\end{verbatim}

\subsection{Logical Operators}
We just saw some logical vectors used in indexing and subsetting. In
general, they will be very useful. The common logical operators are
\texttt{<}, \texttt{>}, \texttt{|}, \texttt{\&}, \texttt{is.*()},
\texttt{!}, \texttt{<=}, \texttt{>=} and \texttt{xor()}.

Two particularly useful functions are \texttt{any()} and
\texttt{each()}. They test a vector of logicals for whether or not at
least one and every element, respectively, evaluates to \texttt{TRUE}.

\subsection{Simple Recoding}
Combining the ways to index particular elements in an object and the
set of logical operators, we are equipped to do some simple recoding
of values within objects that we have already created.

If we find out that Tyson is not from Texas, but from France (hence,
Chatagnier), we will want to change the entry for him in the state
variable. We will leave alone the state variable in the
dataframe. Instead, consider the original vector we created,
\texttt{vState}. If we want Tyson's entry to display that he is from
\texttt{FR} we can use \verb=vState[1] <- "FR"= because we know the
first entry corresponds to him. Confirm that this worked by
considering \texttt{vState}. However, this required we know his entry
was in the first row. If we didn't know this, we could use
\begin{verbatim}
vState[vPeople == "Tyson"] <- "FR"
\end{verbatim}
However, we must be careful. This will change the state to \texttt{FR}
for every Tyson there is. If we have more than one, we'd need to go
about this another way.

Alternatively, we could use the \texttt{ifelse()} function. Suppose we
have to censor the state of origin for all individuals under the age
of 26. We could use
\begin{verbatim}
vState <- ifelse(vAge < 26, NA, vState)
\end{verbatim}
To see why, consider \texttt{?ifelse}.

\section{Summary Functions}
Many times, when using \R{}, we will want to describe many numbers or
observations with some sort of summary value like the mean. We will
use different summaries for categorical, ordinal, and interval level
data. At the same time, we have functions for both univariate and
multivariate summaries.

The functions \texttt{mean()}, \texttt{median()}, \texttt{mode()},
\texttt{sd()}, \texttt{sum()}, \texttt{max()}, and \texttt{min}
describe the central tendencies and other aspects of the data. How
these functions handle missing data (i.e. \texttt{NA}) matters
greatly. Consider the following:
\begin{verbatim}
mean(vVarY)
min(vVarY)
max(vVarY)
sd(vVarY)

mean(vEoM)
mean(vEoM, na.rm = TRUE)
\end{verbatim}

However, instead of evaluating several functions for each vector we
can use \texttt{summary(dfData)} and this will give us most of the
information we might have wanted. \R{} wisely customize the summary
for each variable based on the kind of data it is. For factors and
logicals, we find out about frequencies. For numerical data, we get a
description of the distribution.

Beyond these univariate measures, we can use \texttt{cor()} to
calculate the correlation coefficient between two numerical vectors or
\texttt{table()} to count the number of observations occurring within
each unique classification between multiple factor vectors. Consider
the following:
\begin{verbatim}
cor(vVarY, vVar1)
table(dfData$vAge, dfData$vState)
\end{verbatim}

If we are interested in conditional summaries, we need to do a little
more work. There are two natural ways to approach questions like, what
is the average value of \texttt{vVarY} within people of the same
age. We know there are three different groups because of
\texttt{unique(dfData\$vAge)}. What does \texttt{unique()} do?

We could subset the dataframe according to age, then run
\texttt{summary} or \texttt{mean}. Or, instead, we could use a very
convenient function called \texttt{tapply()}. Consider
\texttt{?tapply} and then the code below.
\begin{verbatim}
mean(subset(dfData, vAge == 25)$vVarY)
tapply(dfData$vVarY, dfData$vAge, mean)
\end{verbatim}

\section{Basic Plots}
Although the numerical/textual summaries of our data will be helpful,
we often look to make figures to communicate some aspect of our date
or some result to others. For now, we will simply go through different
kinds of plots without concerning ourselves with customization.

If we have univariate data, we could use a boxplot or a histogram. Consider
the following:
\begin{verbatim}
boxplot(rnorm(n = 50))
hist(rnorm(n = 50))
\end{verbatim}

If we were so inclined, we could simply plot the values of interest
against an index. Try:
\begin{verbatim}
plot(rnorm(n = 50))
\end{verbatim}

If we have bivariate data, we might want a scatterplot with lines
connecting observations in sequence. Consider:
\begin{verbatim}
plot(rnorm(n = 50), rnorm(n = 50), type = "b")
\end{verbatim}

For three dimensional data, we can generate a surface plot. Use:
\begin{verbatim}
persp(z = matrix(data = sort(rnorm(100, sd = 3)), ncol = 10))
\end{verbatim}

The options are many and
\url{http://cran.r-project.org/web/views/Graphics.html} is a good link
to read. If you know the kind of plot you would like to create, it
will most likely either exist already or you can create it. We will
cover building plots from the ground up and customization later.

\section{Object Bookkeeping}

Recall, just about everything in \R{} is an object. And, by now, we've
created a number of objects. To display all of the objects type
\texttt{ls()}. What is the class of the output of this function call?
If you would like to remove an object named \texttt{foo}, try
\texttt{rm(foo)}. If you would like to remove every object, use
\texttt{rm(list = ls())}. Some folks use this at the top of each
script to make sure their \R{} session is fresh. I would recommend
this.

Now, let's create an object that will contain all of our work. A list
is the best choice here. First, we initialize a list. Second, we will
save the dataframe as one element in this list. Third, and last we
will leave a note for ourselves as a character string.
\begin{verbatim}
lOutput <- list()
lOutput$data <- dfData
lOutput$comment <- "today's work was very rewarding"
lOutput
\end{verbatim}
Now, we can reference the elements in the list by their names and the
\texttt{\$} operator.

To save the object \texttt{lOutput}, first look at
\texttt{?save}. Then, run:
\begin{verbatim}
save(lOutput, file = "Z:/mydata.Rdata"}
\end{verbatim}

%%% Local Variables: 
%%% mode: latex
%%% TeX-master: "../full_course"
%%% End: 

\clearpage
\pagebreak

\part{Case Two: Realistic Data Management and Realistic Plotting}

This part of the short course is designed to take the basic data
management skills acquired in the last part and put them to use on a
more realistic problem. Similarly, we will work with this data to
construct polished figures describing certain aspects of our
data. Lastly, as we proceed through this more realistic example, we
cover topics that you will encounter when you start to work on real
problems, not just problem sets: installing and loading packages,
saving textual output, and saving graphical output.

\par Because some of the code used here is slightly more involved than
an introductory level, don't worry about ``getting it''
immediately. The details of various functions come with time. You can
always refer back to these examples and the \R{} help files when you
do your own work.

\section{Statement of Problem}
Let's consider the United States of America for this example. More
specifically, we will take state level data for 2005. We are
interested in the relationship between the proportion of the
population that is incarcerated, debt and other financial indicators
of the state, and major professional sports teams within the
states. As fascinating of a dataset as this may sound to be, the
reality is that the data does not come pre-loaded in \R{}. Therefore,
we will have to do some work to get the data in \R{} and then to merge
the different sources.

\section{Loading the Data Files}
Although finding usable data is one of the harder components of data
analysis (believe it or not), the files we need for this analysis are
ready and waiting.\footnote{These plain text data files should be
  available however this document was accessed, but, if not, an email
  to \url{jpolmted@gmail.com} will remedy that.}

Locate \texttt{spending.csv}, \texttt{prison.xls}, and
\texttt{teams.csv} on your machine. Ensure that they are in a place
that is easily accessible.

When we load pre-existing data into \R{}, there are two kinds of files
we might work with. If we have data in a plain text format (i.e.\ we
can open it in a text editor) our life is easy (relatively
speaking). If we have data in a binary format (i.e.\ a MS Excel
spreadsheet, a Stata dataset) our life might be easy or it might be
hard. In our case, though, since we have two plain text files and a MS
Excel file (which is standard) we will have no problem getting this
data into \R{}. Indeed, there is an entire manual on this
topic.\footnote{\url{http://cran.r-project.org/doc/manuals/R-data.html}.}

To read in plain text formatted data, try \texttt{?read.table}. There,
we see five related commands. These different functions are based on
the same \texttt{read.table()} function, but have different default
argument values to facilitate reading particular kinds of data. Which
one should we choose? It depends what our data look like. Because
\texttt{spending.csv} and \texttt{teams.csv} are plain text files we
can open them in a text editor. Go ahead and do this. 

We can now see that we are interested in using \texttt{read.csv2()}
because the fields are separated by a semi-colon (\texttt{;}) The
first argument to this function is \texttt{file} and we need to pass
the function a character string describing the location of the file on
the machine. There are two ways of doing this. First, you could give
\R{} the absolute path to the location of the data file.. If you
choose this way, every time you tell \R{} where a file is, you have to
start with a drive (e.g. ``\texttt{C:/}'', ``\texttt{F:/}'') and then
you will work through the hierarchy of directories. Second, you tell
\R{} where your \textit{working directory} is and then, afterwards,
you need only specify the location of the file relative to the working
directory. Let's use the second option. Each way has its advantages
and disadvantages and as you advance, the two will both be easy.

\begin{verbatim}
getwd()
setwd("Z:/stuff")
\end{verbatim}

We can ask \R{} where the current working directory was and then we
change it to the directory \texttt{Z:/stuff}. Be certain that the
directory exists. What happens if it doesn't?  Also, make sure the
data files are in this directory.

Try:
\begin{verbatim}
read.csv2(file = "teams.csv", header = FALSE, skip = 5)
\end{verbatim}

If we use one of the other functions, \R{} will get structure the data
incorrectly (unless we override all of the defaults). Test it out to
verify this. 

What is the output of this call to \texttt{read.csv2()} (Hint: use
\texttt{is})? What kind of object is the returned value from this
function? What do the arguments we passed do?

This time, let's use
\begin{verbatim}
dfTeams <- read.csv2(file = "teams.csv", header = FALSE, skip = 5)
dfSpending <- read.csv2(file = "spending.csv", header = TRUE, skip = 7)
\end{verbatim}

The second \textit{comma separated values} file is set up slightly
differently so we need to change the arguments.

The last file we need to load in is an MS Excel file. Take a moment to
try to find a function to read in MS Excel files. The package
\texttt{gdata} provides the function \texttt{read.xls()}. To access
this functionality, we will install the package, load it, and then
read the help file.

\begin{verbatim}
install.packages("gdata")
library("gdata")
?read.xls
\end{verbatim}

After looking at the help file, try
\begin{verbatim}
dfPrison <- read.xls("prison.xls", skip = 3, header = TRUE)
\end{verbatim}

As with the first two \texttt{read.*} functions, if we do not assign
the object to an identifier the data are read into \R{} and the
dataframe object is printed to the display. By assigning it, we can
refer the dataframe by name instead of re-importing it.

\section{Cleaning and Merging the Data Files}
Take a look at the dataframes we have created. If the output is a
little overwhelming, try using the \texttt{head()} and \texttt{tail()}
commands. How many observations are there in each dataframe? How many
variables?
\begin{verbatim}
dfSpending
dfTeams
dfPrison

head(dfSpending)
head(dfTeams)
head(dfPrison)

tail(dfSpending)
tail(dfTeams)
tail(dfPrison)

dim(dfSpending)
dim(dfTeams)
dim(dfPrison)
\end{verbatim}
Notice that \texttt{dfSpending} has more than 50 rows. Look at the
dataframe. Why is this? Notice that \texttt{dfTeams} has only 26
rows. The other states not listed have 0 teams. We aren't ``missing''
the data, per se. It just isn't included. However, for
\texttt{dfPrisons} we don't have data for two of the states. Each of
these issues needs to be treated in turn so that we can produce a
unified, homogeneous dataset.

Clean up the spending data with the following code:
\begin{verbatim}
dfSpending1 <- dfSpending[-c(9, 45),]
\end{verbatim}
What are we doing here?

In order to merge these dataframes together we will use the function
\texttt{merge()}. Open the help file. Let's start with merging the
spending data and the sports teams data. Then, we'll clean up the
variable names and address the ``missing'' data for states without any
teams. 

\begin{verbatim}
dfData1 <- merge(dfSpending1, dfTeams, by.x = "State", by.y = "V1", all.x = TRUE)
names(dfData1)[names(dfData1) == "V2"] <- "teams"
names(dfData1)[1:6] <- c("state", "sls", "sld", "gsp", "rsg", "pop")
dfData1$teams[is.na(dfData1$teams)] <- 0
\end{verbatim}

\begin{itemize}
\item we merge two dataframes by specifying which variable in each
  dataframe should be used as a key
\item the merged dataframe is assigned to a new identifier
\item we manipulate the column names of the dataframe by working on
  the vector object that results from \texttt{names()}
\item we force the value of the variable to be 0 for any observation
  where we previously had it coded as \texttt{NA}.
\end{itemize}

The last step is important because we know these states have 0
teams. That is different from the missingness implied by an
\texttt{NA}. Observations with \texttt{NA}'s are often dropped, so
replacing these with 0 is a good thing for our application.

In order to merge \texttt{dfData1} with \texttt{dfPrison}, we would
need them to have a variable in common and they don't. The closest we have are the
state variables. But, they are not identical, so the merge would not
result in what we want. First, we code a new state variable in
\texttt{dfData1} and then we merge the dataframes.
\begin{verbatim}
dfData1$state_lower <- tolower(as.character(dfData1$state))
dfData2 <- merge(dfData1, dfPrison, by.x = "state_lower", by.y = "state", all = TRUE)
\end{verbatim}
This is the second block of code that manipulates character vectors
for some purpose. This will become a useful skill.

Finally, we have a dataframe with all of our variables in it. Let's
add one more variable based on whether or not the state is in the
south.
\begin{verbatim}
vSouth <- c("arizona", "new mexico", "texas", "oklahoma", "arkansas",
            "louisiana", "mississippi", "alabama", "florida", "georgia",
            "tennesee", "south carolina", "north carolina", "kentucky",
            "virginia", "west virginia"
            )
dfData2$south <- is.element(dfData2$state_lower, vSouth)
\end{verbatim}

The function \texttt{is.element()} is a helpful function. Type
\texttt{?set} to see similar functions. This is a clear example of one
issue many \R{} beginners struggle with: \textit{finding the name of
  the function that performs fairly straightforward tasks}. As you use
\R{} more and more, you come across an increasing number of these
functions and will recall them in the future. Until then, don't
hesitate to ask.

In this code, I am just tidying up some vector types and coercing my
way through. You need not understand exactly what is going on
here. You will in time, though.
\begin{verbatim}
dfData2[, 3:7] <- apply(dfData2[, 3:7], 2, function(X) as.numeric(as.character(X)))
\end{verbatim}

This has been a good amount of work to create a unified dataframe from
disparate sources. Moreover, this data started off in a fairly clean
form.\footnote{I did some pre-processing of the data to simplify steps
  in \R{}.} In practice, you will spend a lot of time getting your
data ready for analysis. Some people prefer to do these steps in other
software packages like Stata. This is wholly unnecessary and is not
recommended. Everything can be done in \R{}.

\section{Numerical Summaries of the Data}

We have previously covered the use of simple functions to provide
quantitative summaries of our data. So, for this example, instead of
repeating those steps, we will focus on some more advanced functions.

In the future, the packages \texttt{reshape} and \texttt{plyr} should
be used early and often. They provide some of the smartest functions
related to data management in \R{}.
\begin{verbatim}
install.packages(c("reshape", "plyr"))
library(reshape)
library(plyr)
\end{verbatim}

We recall that \texttt{summary} gave us univariate descriptions of
each of our variables. For example, there are fifteen states in the
south (\texttt{south}), the median state population is 4.2 million
(\texttt{pop}), and the maximum real state growth rate is 9.4\%
(\texttt{rsg}). Instead, suppose we wanted to discuss these same
quantities, but we wanted to summarize the data as if it belonged to
two different groups, those states that have professional sports teams
and those that do not.

We could subset the data twice and then invoke
\texttt{summary}. However, there are cleaner and more powerful ways to
do this. Consider this code:
\begin{verbatim}
dfData2$teams2 <- ifelse(dfData2$teams > 0, 1, 0)
dfTemp1 <- melt(dfData2, measure.vars = c("rsg", "pop"), 
                         id.vars = c("state", "south", "teams2")
                )
cast(dfTemp1, teams2 ~ ., subset = (variable == "rsg"), mean, margins = TRUE)
\end{verbatim}
We have calculated the mean real state growth rate for those states
with professional sports teams and those without.

Similarly, we can calculate the average population for states falling
within the four categories obtained when we cross-classify being in
the south with having a professional sports team. The average state
population is larger in states in the south than those not in the
south. Similarly, the average population in states with sports teams
is larger than those without.

Although the dataset we put together contains all of the information
to compute these values, their calculation is not immediate with any
of the tools we yet know.

Lastly, consider the function \texttt{sink()}. Rerun the
\texttt{melt()} and \texttt{cast()} code after trying:
\begin{verbatim}
sink(file = "foo.txt")
\end{verbatim}
All of the \R{} output will be sent to that file instead of the
screen. After running the aggregation, run \texttt{sink()} to close
the connection. Now, open up \texttt{foo.txt}. If you add the line
\texttt{cat("output")} in with your code it will annotate the
``sunken'' output. Still, you must remember that you will not have the
luxury of knowing exactly which command generated what unless you
compare it to your source code.

\section{Graphical Summaries of the Data}

Now, we will look at how to create a polished plot and save it to an
external file. There is no way to introduce all of the possible plot
types that are available. The way to learn this is to focus on
understanding what the plotting parameters mean. Then, at a later
date, you can combine that knowledge, your experience, and the \R{}
help files together to figure it out. The \texttt{plot()} command is
very generic and will try to guess the plot that makes the most sense
for your data.

Create a scatter plot of \textit{number of professional teams} against
\textit{state population}.
\begin{verbatim}
plot(dfData2$pop, dfData2$teams)
\end{verbatim}

Load the \texttt{ggplot2} package (or install it first if it is not
installed). We want the \texttt{alpha()} function from the
\texttt{ggplot2} package and the below code will not work without it.

The previous plot is rather spartan and not very helpful. Instead, try
\begin{verbatim}
plot(x = subset(dfData2, south == 0)$pop, y = subset(dfData2, south == 0)$teams,
     main = "Teams by Population",
     xlab = "State Population (Millions of People)",
     ylab = "Professional Sports Teams (Count of Teams for Real Sports)",
     cex.main = 2,
     pch = 3,
     col = 3
     )
points(x = subset(dfData2, south == 1)$pop, 
       y = subset(dfData2, south == 1)$teams,
       pch = 1,
       col = 5
       )
abline(v = mean(dfData2$pop),
       col = alpha(1, .3)
       )
abline(h = mean(dfData2$teams),
       col = alpha(1, .3)
       )
text(x = 15, y = 17,
     paste("Correlation", cor(dfData2$pop, dfData2$teams))
     )
legend(x = 20, y = 15,
       legend = c("Non-South", "South", "", "Marginal Avg"),
       pch = c(3, 1, NA, NA),
       col = c(3, 5, NA, alpha(1, .3)),
       lty = c(NA, NA, NA, 1)
       )
\end{verbatim}
This figure is produced using what is known as the \textit{base}
graphics package.
\begin{itemize}
\item \texttt{plot()} creates the initial scatter plot based on the
  first subset of the data
\item since \texttt{plot()} is deciding on the geometry of the figure,
  it will reflect only the data that it has
\item overriding the automatic calculation of axis limits and other
  aspects is sometimes the trick to successfully plotting multiple
  instances of subsetted data on the same figure
\item we set the title (\texttt{main}) and labels (\texttt{xlab},
  \texttt{ylab}) arguments here
\item \texttt{points()} allows us to add points to an already existing
  plot (like the one we just created with \texttt{plot()})
\item \texttt{abline()} is a general function that allows us to add
  straight lines
\item notice that when we specify the color (\texttt{col = ...}) we
  can use the \texttt{alpha()} function to provide us
  transparency\footnote{For more help on choosing colors see
    \url{http://research.stowers-institute.org/efg/R/Color/Chart/}}
\item \texttt{text()} allows us to add arbitrary text
\item instead of hard coding the correlation coefficient we used the
  \texttt{paste()} command to turn a string and a numeric object into
  one longer string
\item \texttt{legend()} allows us to add a legend where we want
  (\texttt{x, y})
\item the legend is entirely custom, so it reflects only what you tell
  it and the order in which you tell it things
\item we customize the legend by specifying which shape, color,
  linetype, etc. should show up on each line, \texttt{NA} let's us
  ``skip'' one of these attributes
\end{itemize}

Alternatively, we could compare distributions of the incarceration
rates among states for states in the south vs.\ those not in the
non-south and for whites vs.\ blacks.

\begin{verbatim}
boxplot(value  ~ variable + south,
        data = melt(dfData2, id.vars = c("south", "state"),
            measure.vars = c("white", "black")            
            ),
        notch = FALSE,
        names = c("White, Non-South", "Black, Non-South",
            "White, South", "Black, South"
            ),
        main = "State Incarceration Rates by Region and Race",
        ylab = "Count Incarcerated per 100,000 Population"
        )
\end{verbatim}

We can see several things: one \R{} related, one not. First, some of
the specifics change depending on the kind of plot we are making, but
many things carry over directly. Second, it is no surprise this
country has some race relation problems. We can't even blame the south
in this instance! (Don't close the plot.)

In fact, this reality is so unbearable that you should type:
\begin{verbatim}
rect(0,0,5,5000,
     density = 1,
     angle = 135,
     lwd = 10
     )
rect(0,0,5,5000,
     density = 1,
     angle = 45,
     lwd = 10
     )
title(sub = "lock this result up!")
\end{verbatim}
The point of this demonstration is to show you that the order in which
you add things to plots matters. The items at the end are layered on
top of the previous parts. Whatever you want on the bottom is what you
should plot first.

Also, notice that coordinates given to specify locations are relative
to the units on the axes of the figure.

Now, there are two ways to save these figures. There is the
\textit{smart} way and then the other way. The other way involves
using the menu and saving the image as some inferior file format. Do
not do this.

Instead, consider the figure created by
\begin{verbatim}
plot(density(replicate(1000, mean(runif(10))
                       ),
             kernel = "rectangular"
             )
     )
\end{verbatim}

You can save this file as a \texttt{.pdf} with 
\begin{verbatim}
pdf(file = "draws.pdf")

plot(density(replicate(1000, mean(runif(10))
                       ),
             kernel = "rectangular"
             )
     )

dev.off()
\end{verbatim}
This figure is saved as a \texttt{.pdf} file to the working
directory. And, in fact, everything I add to this plot will continue
to go to the file until the command \texttt{dev.off()}. Note, this
command is necessary to finalize the file. Similar functions exist to
save figures as other filetypes: \texttt{jpeg()}, \texttt{ps()},
\texttt{png()}.

In the long run, I recommend using the \texttt{ggplot2} package to
create figures and not \textit{base}, though \textit{base} is a good
place to start. Not surprisingly, \texttt{ggplot2} is maintained by
the same person who authored the \texttt{reshape} and \texttt{plyr}
packages. Also, you may be interested in the \texttt{tikzDevice}
package for inclusion of figures natively in \LaTeX{}.



%%% Local Variables: 
%%% mode: latex
%%% TeX-master: "../full_course"
%%% End: 


\clearpage
\pagebreak

\part{Writing Programs}
\section{Control Flow Operations}
\subsection{Conditionals}

We often want to check a conditional statement and then do something
in response to that conditional holding, or not holding. Try (writing
in RWinEdt):
\begin{verbatim}
d <- runif(1)
if (d > 0.5) {
  cat("\n\n today is a good day \n\n")
}
\end{verbatim}

Here is what is happening:

\begin{enumerate}
\item \texttt{d} is a random uniform number drawn from (0,1)%
\item we tell \R: if \texttt{d} is greater than 0.5, then do the
  operation in the curly \texttt{} braces.%
\item the operation in the curly braces is a \texttt{cat} statement
  (short for `concatenate and print') which will print the contents of
  the quotation marks.%
\item the \verb=\n= are simply line breaks to impose a bit of
  space between the prompt and our output.  Here that means two breaks
  either side of the text.
\end{enumerate}

Perhaps we need \R{} to do something if the condition isn't met.  No
problem (note the curly braces!):
\begin{verbatim}
if(runif(1)>0.5){
  cat("\n\n the beatles are on itunes\n\n")
} else {
  cat("\n the beatles are of questionable skill \n")
}
\end{verbatim}
A simpler, `hard wired' alternative to this is \texttt{ifelse()} and
we've already encountered this.

Suppose we want to check more than one condition. Then \verb!&! will
come in handy:
\begin{verbatim}

x <- "3"
y <- cos(3)

if (runif(1) > 0.5 & rnorm(1) < 0) {
  print(x); print(y)
} else{
  (plot(density(rnorm(100))))
}

if (!(rnorm(1) > 0) | !(runif(1) < 0.5)) {
  print(x)
  print(y)
} else{
  (plot(density(rnorm(100))))
}
\end{verbatim}
Notice the use of \texttt{;} to have \R{} do a couple of things.
Sometimes you'll see the use of \verb!&&!.  This means that the second
conditional is only checked if the first one is true.  This
would make no difference to the example above. See the help file for more information.

Usefully, we can nest \texttt{if()} loops.  Try the following:
\begin{verbatim}

if (r<-runif(1) > 0.5){
  cat("r is",r,"\n")
  ##  
  if(r < .6){
    print("0.5 to 0.6")
  } else {
    if(.6 < r & r < .7) {
      print("0.6 to 0.7")
    } else {
      print("bigger than 0.7")
    }
  }
}  else {
  print("smaller than 0.5")
}

\end{verbatim}

Lots of things to notice here:
\begin{enumerate}
\item \texttt{(r <- runif(1)) > 0.5} checks the conditional \textit{and}
assigns the number to \texttt{r} in one go.

\item \verb!{cat("r is",r,"\n")}! reports back the value actual
value  of \texttt{r} that has been assigned (notice the use of the
commas) \textit{outside} the quotation marks.

\item \texttt{if(r < .6){print("0.5 to 0.6")}} this occurs conditional
on \texttt{r} being greater than .5, but less than .6

\item \verb!else if (.6 < r & r < .7) {print("0.6 to 0.7")}! is checked if
\texttt{r} is greater than 0.5, and it is not less than 0.6.

\item \verb!else {print("bigger than 0.7")}! prints something if
the previous statements are false (but \texttt{r>}0.5)

\item \verb!else {(print("smaller than 0.5"))}! is the last line of
the program and matches the first conditional (i.e. we are now in
the case where \texttt{r} is less than 0.5)
\end{enumerate}

\subsection{Loops}
Loops, like crystal methamphetamine, are easy to use, and easy to
abuse. They are the workhorses of much of the \R that gets written,
partly because they are so straightforward to write.  This is
unfortunate is some ways, because they are often inefficient.
Throwing caution to the wind, try the following in:

\begin{verbatim}
for (i in 1 : 1000) {
  hist(rnorm(100), col = i, main = paste("Picture", i, sep = " "))
}
\end{verbatim}

\begin{enumerate}
\item this is a \texttt{for} loop: the give away is in the first
line.  We are saying, `for' \texttt{i} between 1 and 1000, do the
thing in the curly braces.

\item here that is: draw a histogram of 100 random normal points,
color it with color number \texttt{i} and then call it
\texttt{Picture i} where \texttt{i}, of course, is just a number.

\item actually there are not 1000 colors in \R{}'s palette, so it
recycles some.

\item see \texttt{?paste} to read the way it is used
\end{enumerate}

Of course, from a programming perspective, having this loop run
every time you run the program maybe annoying.  One option is simply
to wrap it into a function.  So:
\begin{verbatim}
homework<-function(){
  for(i in 1:1000){
    hist(rnorm(100), col = i,
         main = paste("Picture", i, sep = " ")
         )
  }
}
\end{verbatim}
Which means that the \texttt{for} loop won't run until we call it
via \texttt{homework()}

Most of the time, we want to loop through a matrix (or data set)
take something from that matrix and put it somewhere else.

First off, create a matrix---initially filled with missing
values---to take the fruit of our labors:
\begin{verbatim}
mOutput <- matrix(NA, nrow = 30 , ncol = 1)
\end{verbatim}
Now, suppose we have a matrix like this

\begin{verbatim}
mData <- matrix(runif(900), nrow = 30)
\end{verbatim}

That is, $30\times30$ with 900 random numbers.  And we want to go
row by row taking the mean of the row and outputting it.  This would
work:
\begin{verbatim}
for (i in 1 : nrow(mData)) {
  mOutput[i] <- mean(mData[i, ])
}
\end{verbatim}

Notice:
\begin{enumerate}
\item we can use pretty much anything for our index: here it is
  \texttt{i}; if we replaces every reference to \texttt{i} in this
  expression with \texttt{monkey}, that would work too

\item the end of the index is the number of rows in \texttt{mData}
  (which is 30)

\item now, we are assigning the mean of the $i^{th}$ row of
\texttt{mData} to the $i^{th}$ row of \texttt{mOutput}
\end{enumerate}

So what's the problem?  The loop is perfectly accurate, but it is
slow, and laborious to code. 

We may sometimes have cause to place \texttt{for} loops within other
\texttt{for} loops (but generally try to avoid). Here is an example:
\begin{verbatim}

mLetters <- matrix(data = NA, nrow = 10, ncol = 5)
for (m in 1:10) {
  for(n in 1:5){
    mLetters[m,n] <- (letters[n + m])
  }
}
\end{verbatim}

Look at \texttt{letters}. What is it? What does this code do?

Notice that the way we are indexing the loops
matters here.  If we put \texttt{mLetters[n, m]} instead of
\texttt{mLetters[m, n]}in our code, we'll get the dreaded
\begin{verbatim}
Error: subscript out of bounds
\end{verbatim}
There are alternatives to \texttt{for} which are used in different
circumstances.  Examples are \texttt{while} and \texttt{repeat}
which are often used together.

\subsection*{Other Types of Loops}
The syntax for \texttt{while} is \texttt{while(condition)
expression} which means while a particular condition holds, the
\texttt{expression} will be executed. \texttt{repeat(expression)}
simply repeats the \texttt{expression} operation again and again and
again.  This type of thing turns up a lot in monte-carlos. Consider
the following:
\begin{verbatim}
n <- 1
mF <- matrix(NA, nrow = 10, ncol = n)

while (n < 36) {
  repeat{
    vF <- rbinom(10, 1, 0.5)
    if (sum(vF) %% 2 == 0) {
      break()
    }
  }
  mF <- matrix(cbind(mF[, 1 : (n-1)], vF), nrow = 10, ncol = n)
  n<-n + 1
}
\end{verbatim}
Here, while \texttt{n} is less than 36, I need to repeatedly sample
from a binomial (with a sample size of 10) until an even number of
the sample are ones.  So, for example, $[1,0,0,0,0,0,0,1,1,0]$ won't
do, but $[1,1,0,0,1,1,0,1,1,0]$ is fine. As soon as I get a sample
fulfilling my requirements, it should be stored (in \texttt{mF}).
Notice that I have to increment the loop with \texttt{n + 1} or else
the condition \texttt{n < 36} will be true forever, and the program
will never stop.

Just to make this point, consider
\begin{verbatim}
repeat(cat("\n lake effect snow today\n"))
\end{verbatim}
which won't end until you hit \texttt{esc} or \texttt{STOP}.

\section{Sampling}
Quite often we have to sample from a (posterior) distribution we
created. In general, we want to sample \texttt{n} values, but we want
the sample we produce to be `weighted'---in the sense that it is
proportional to the mass---for each value of our discrete support
(say, the thousand values of $\theta$ we created for our homework.

We could put together the numerical cdf, but it easier to use
\texttt{sample} directly.
\begin{verbatim}
vCand <- seq(0, 1, 0.001)
vP <- dchisq(vCand, df = 15)
vS1 <- sample(vCand, 1000, replace = TRUE, prob = vP)

vDraws <- rnorm(length(vCand))
vS2 <- sample(vDraws, length(vDraws), replace = TRUE)
\end{verbatim}

Here, I'm assuming that we first took 1000 values between zero and
one, and then we worked out that the posterior,$\Pr(\theta|y)$, was
$\chi^2_{15}$ (which is unlikely, but anyway).  Then we sampled:
\begin{enumerate}
\item the object of our sampling was our candidate $\theta$ values.%
\item we wanted a sample of size 1000
\item we \texttt{replace}d the candidates each time we sampled
\item we set \texttt{prob = vP} which means we are sampling in
proportion to the posterior we calculated.
\item What could \texttt{vS2} look like if we used \texttt{replace = FALSE}

\end{enumerate}

\section{Functional Programming}
\subsection{Basics}
Creating a function is very easy and any time you use code over and
over again (e.g.\ typed manually, in a loop, next semester) you can
and should wrap that code up in a function. Functions take in
arguments (objects) and return a single object.

Consider:
\begin{verbatim}
SayHello <- function()
  {
    cat("\n \n \n Hello World \n \n \n")
  }
\end{verbatim}
Try both \texttt{SayHello} and \texttt{SayHello()}. One displays the
function itself and one displays the output. That is, loosely
speaking, the two are the difference between $f$ and $f(x)$ for some
fixed value $x$.

In this function, there are no arguments and we don't create any
objects for latter use, but this need not be the case.

\begin{verbatim}
GetCos <- function (x = 0)
  {
    vC <- cos(x)

    return(vC)
  }
\end{verbatim}

\texttt{GetCos()} takes a single argument which has a default value of
0. Then, we calculate the cosine of $x$ and assign it to
\texttt{vC}. Lastly, before the function ends, we tell \R{} that the
output of the function is \texttt{vC}.

Try \texttt{GetCos(1)} and \texttt{GetCos()}. Have we created any new
objects in our workspace, namely \texttt{vC}? No. To save the output
for latter use we would do
\begin{verbatim}
vOut <- GetCos()
\end{verbatim}

If we have more than one object to return at the end of the function,
we can wrap these objects up together in another single object.

Consider 
\begin{verbatim}
GetTan <- function (x)
  {
    vT <- tan(x)

    vTT <- "your face is a tangent!"
    
    return(list(vT, vTT
                )
           )
  }

GetTan(1)
\end{verbatim}

We return the output as a list (because the objects had different
modes) and we use something like \verb=lOutput2 <- GetTan(4)= to save
it for future use.

If you pass a function an argument that it was not expecting based on the
function definition, you will get an error.

If you don't pass a function an argument that it was expecting based
on the function definition, you will get an error.

\subsection{Scope}
There are some very important rules to follow when using functions to
keep things clear. First, realize that a function is a temporary
workspace (or environment in \R{}-speak). The arguments are one-way
traffic onto this island. The \texttt{return} statement is one-way
traffic off the island.
\begin{enumerate}
\item any object not created by and contained within your function
  should be passed to it as an argument

\item any object(s) to be used later should be returned by your
  function

\item functions (with the exceptions of \texttt{save}, \texttt{plot},
  etc.) should not have side-effects
\end{enumerate}

Consider this code:
\begin{verbatim}
vThing <- 4

GetSquare <- function()
  {
    vThing ^ 2
  }

GetSquare()

rm(vThing)

GetSquare()
\end{verbatim}

We violated our rule of relying on other objects only through the
arguments. And, we can see how easily our function becomes worthless
if we don't listen to that rule.

Using the \verb!<<-! operator is a quick way to violate our other
rules.

Try:
\begin{verbatim}
BadFunc <- function()
  {
    cat("i woke up at 8am and all i got was this lousy R function")
    
    mLetters <<- "gone! mwahhahhah"
  }
\end{verbatim}

Look at \texttt{mLetters} from before. We changed it even though we
didn't pass the function any arguments, we didn't return any object as
output, and we didn't assign output to an identifier. Had we used
\verb=<-= instead of \verb=<<-=, we would only have altered the object
\texttt{mLetters} within the function environment so nothing would
have been overwritten.




%%% Local Variables: 
%%% mode: latex
%%% TeX-master: "../full_course"
%%% End: 


\clearpage



%% \bibliographystyle{}
%% \bibliography{}

\end{document}

%%%%%%%%%%%%%%%%%%%%%%%%%%%%%%%%%%%%%%%%%%%%%%%%%%%%%%%%%%%%%%%%%%%%%%%%%%%%%%% 
%%%%%%%%%%%%%%%%%%%%%%%%%%%%%%%%%%%%%%%%%%%%%%%%%%%%%%%%%%%%%%%%%%%%%%%%%%%%%%% 
%%%%%%%%%%%%%%%%%%%%%%%%%%%%%%%%%%%%%%%%%%%%%%%%%%%%%%%%%%%%%%%%%%%%%%%%%%%%%%% 



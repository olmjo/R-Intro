\part{Writing Programs}
\section{Control Flow Operations}
\subsection{Conditionals}

We often want to check a conditional statement and then do something
in response to that conditional holding, or not holding. Try (writing
in RWinEdt):
\begin{verbatim}
d <- runif(1)
if (d > 0.5) {
  cat("\n\n today is a good day \n\n")
}
\end{verbatim}

Here is what is happening:

\begin{enumerate}
\item \texttt{d} is a random uniform number drawn from (0,1)%
\item we tell \R: if \texttt{d} is greater than 0.5, then do the
  operation in the curly \texttt{} braces.%
\item the operation in the curly braces is a \texttt{cat} statement
  (short for `concatenate and print') which will print the contents of
  the quotation marks.%
\item the \verb=\n= are simply line breaks to impose a bit of
  space between the prompt and our output.  Here that means two breaks
  either side of the text.
\end{enumerate}

Perhaps we need \R{} to do something if the condition isn't met.  No
problem (note the curly braces!):
\begin{verbatim}
if(runif(1)>0.5){
  cat("\n\n the beatles are on itunes\n\n")
} else {
  cat("\n the beatles are of questionable skill \n")
}
\end{verbatim}
A simpler, `hard wired' alternative to this is \texttt{ifelse()} and
we've already encountered this.

Suppose we want to check more than one condition. Then \verb!&! will
come in handy:
\begin{verbatim}

x <- "3"
y <- cos(3)

if (runif(1) > 0.5 & rnorm(1) < 0) {
  print(x); print(y)
} else{
  (plot(density(rnorm(100))))
}

if (!(rnorm(1) > 0) | !(runif(1) < 0.5)) {
  print(x)
  print(y)
} else{
  (plot(density(rnorm(100))))
}
\end{verbatim}
Notice the use of \texttt{;} to have \R{} do a couple of things.
Sometimes you'll see the use of \verb!&&!.  This means that the second
conditional is only checked if the first one is true.  This
would make no difference to the example above. See the help file for more information.

Usefully, we can nest \texttt{if()} loops.  Try the following:
\begin{verbatim}

if (r<-runif(1) > 0.5){
  cat("r is",r,"\n")
  ##  
  if(r < .6){
    print("0.5 to 0.6")
  } else {
    if(.6 < r & r < .7) {
      print("0.6 to 0.7")
    } else {
      print("bigger than 0.7")
    }
  }
}  else {
  print("smaller than 0.5")
}

\end{verbatim}

Lots of things to notice here:
\begin{enumerate}
\item \texttt{(r <- runif(1)) > 0.5} checks the conditional \textit{and}
assigns the number to \texttt{r} in one go.

\item \verb!{cat("r is",r,"\n")}! reports back the value actual
value  of \texttt{r} that has been assigned (notice the use of the
commas) \textit{outside} the quotation marks.

\item \texttt{if(r < .6){print("0.5 to 0.6")}} this occurs conditional
on \texttt{r} being greater than .5, but less than .6

\item \verb!else if (.6 < r & r < .7) {print("0.6 to 0.7")}! is checked if
\texttt{r} is greater than 0.5, and it is not less than 0.6.

\item \verb!else {print("bigger than 0.7")}! prints something if
the previous statements are false (but \texttt{r>}0.5)

\item \verb!else {(print("smaller than 0.5"))}! is the last line of
the program and matches the first conditional (i.e. we are now in
the case where \texttt{r} is less than 0.5)
\end{enumerate}

\subsection{Loops}
Loops, like crystal methamphetamine, are easy to use, and easy to
abuse. They are the workhorses of much of the \R that gets written,
partly because they are so straightforward to write.  This is
unfortunate is some ways, because they are often inefficient.
Throwing caution to the wind, try the following in:

\begin{verbatim}
for (i in 1 : 1000) {
  hist(rnorm(100), col = i, main = paste("Picture", i, sep = " "))
}
\end{verbatim}

\begin{enumerate}
\item this is a \texttt{for} loop: the give away is in the first
line.  We are saying, `for' \texttt{i} between 1 and 1000, do the
thing in the curly braces.

\item here that is: draw a histogram of 100 random normal points,
color it with color number \texttt{i} and then call it
\texttt{Picture i} where \texttt{i}, of course, is just a number.

\item actually there are not 1000 colors in \R{}'s palette, so it
recycles some.

\item see \texttt{?paste} to read the way it is used
\end{enumerate}

Of course, from a programming perspective, having this loop run
every time you run the program maybe annoying.  One option is simply
to wrap it into a function.  So:
\begin{verbatim}
homework<-function(){
  for(i in 1:1000){
    hist(rnorm(100), col = i,
         main = paste("Picture", i, sep = " ")
         )
  }
}
\end{verbatim}
Which means that the \texttt{for} loop won't run until we call it
via \texttt{homework()}

Most of the time, we want to loop through a matrix (or data set)
take something from that matrix and put it somewhere else.

First off, create a matrix---initially filled with missing
values---to take the fruit of our labors:
\begin{verbatim}
mOutput <- matrix(NA, nrow = 30 , ncol = 1)
\end{verbatim}
Now, suppose we have a matrix like this

\begin{verbatim}
mData <- matrix(runif(900), nrow = 30)
\end{verbatim}

That is, $30\times30$ with 900 random numbers.  And we want to go
row by row taking the mean of the row and outputting it.  This would
work:
\begin{verbatim}
for (i in 1 : nrow(mData)) {
  mOutput[i] <- mean(mData[i, ])
}
\end{verbatim}

Notice:
\begin{enumerate}
\item we can use pretty much anything for our index: here it is
  \texttt{i}; if we replaces every reference to \texttt{i} in this
  expression with \texttt{monkey}, that would work too

\item the end of the index is the number of rows in \texttt{mData}
  (which is 30)

\item now, we are assigning the mean of the $i^{th}$ row of
\texttt{mData} to the $i^{th}$ row of \texttt{mOutput}
\end{enumerate}

So what's the problem?  The loop is perfectly accurate, but it is
slow, and laborious to code. 

We may sometimes have cause to place \texttt{for} loops within other
\texttt{for} loops (but generally try to avoid). Here is an example:
\begin{verbatim}

mLetters <- matrix(data = NA, nrow = 10, ncol = 5)
for (m in 1:10) {
  for(n in 1:5){
    mLetters[m,n] <- (letters[n + m])
  }
}
\end{verbatim}

Look at \texttt{letters}. What is it? What does this code do?

Notice that the way we are indexing the loops
matters here.  If we put \texttt{mLetters[n, m]} instead of
\texttt{mLetters[m, n]}in our code, we'll get the dreaded
\begin{verbatim}
Error: subscript out of bounds
\end{verbatim}
There are alternatives to \texttt{for} which are used in different
circumstances.  Examples are \texttt{while} and \texttt{repeat}
which are often used together.

\subsection*{Other Types of Loops}
The syntax for \texttt{while} is \texttt{while(condition)
expression} which means while a particular condition holds, the
\texttt{expression} will be executed. \texttt{repeat(expression)}
simply repeats the \texttt{expression} operation again and again and
again.  This type of thing turns up a lot in monte-carlos. Consider
the following:
\begin{verbatim}
n <- 1
mF <- matrix(NA, nrow = 10, ncol = n)

while (n < 36) {
  repeat{
    vF <- rbinom(10, 1, 0.5)
    if (sum(vF) %% 2 == 0) {
      break()
    }
  }
  mF <- matrix(cbind(mF[, 1 : (n-1)], vF), nrow = 10, ncol = n)
  n<-n + 1
}
\end{verbatim}
Here, while \texttt{n} is less than 36, I need to repeatedly sample
from a binomial (with a sample size of 10) until an even number of
the sample are ones.  So, for example, $[1,0,0,0,0,0,0,1,1,0]$ won't
do, but $[1,1,0,0,1,1,0,1,1,0]$ is fine. As soon as I get a sample
fulfilling my requirements, it should be stored (in \texttt{mF}).
Notice that I have to increment the loop with \texttt{n + 1} or else
the condition \texttt{n < 36} will be true forever, and the program
will never stop.

Just to make this point, consider
\begin{verbatim}
repeat(cat("\n lake effect snow today\n"))
\end{verbatim}
which won't end until you hit \texttt{esc} or \texttt{STOP}.

\section{Sampling}
Quite often we have to sample from a (posterior) distribution we
created. In general, we want to sample \texttt{n} values, but we want
the sample we produce to be `weighted'---in the sense that it is
proportional to the mass---for each value of our discrete support
(say, the thousand values of $\theta$ we created for our homework.

We could put together the numerical cdf, but it easier to use
\texttt{sample} directly.
\begin{verbatim}
vCand <- seq(0, 1, 0.001)
vP <- dchisq(vCand, df = 15)
vS1 <- sample(vCand, 1000, replace = TRUE, prob = vP)

vDraws <- rnorm(length(vCand))
vS2 <- sample(vDraws, length(vDraws), replace = TRUE)
\end{verbatim}

Here, I'm assuming that we first took 1000 values between zero and
one, and then we worked out that the posterior,$\Pr(\theta|y)$, was
$\chi^2_{15}$ (which is unlikely, but anyway).  Then we sampled:
\begin{enumerate}
\item the object of our sampling was our candidate $\theta$ values.%
\item we wanted a sample of size 1000
\item we \texttt{replace}d the candidates each time we sampled
\item we set \texttt{prob = vP} which means we are sampling in
proportion to the posterior we calculated.
\item What could \texttt{vS2} look like if we used \texttt{replace = FALSE}

\end{enumerate}

\section{Functional Programming}
\subsection{Basics}
Creating a function is very easy and any time you use code over and
over again (e.g.\ typed manually, in a loop, next semester) you can
and should wrap that code up in a function. Functions take in
arguments (objects) and return a single object.

Consider:
\begin{verbatim}
SayHello <- function()
  {
    cat("\n \n \n Hello World \n \n \n")
  }
\end{verbatim}
Try both \texttt{SayHello} and \texttt{SayHello()}. One displays the
function itself and one displays the output. That is, loosely
speaking, the two are the difference between $f$ and $f(x)$ for some
fixed value $x$.

In this function, there are no arguments and we don't create any
objects for latter use, but this need not be the case.

\begin{verbatim}
GetCos <- function (x = 0)
  {
    vC <- cos(x)

    return(vC)
  }
\end{verbatim}

\texttt{GetCos()} takes a single argument which has a default value of
0. Then, we calculate the cosine of $x$ and assign it to
\texttt{vC}. Lastly, before the function ends, we tell \R{} that the
output of the function is \texttt{vC}.

Try \texttt{GetCos(1)} and \texttt{GetCos()}. Have we created any new
objects in our workspace, namely \texttt{vC}? No. To save the output
for latter use we would do
\begin{verbatim}
vOut <- GetCos()
\end{verbatim}

If we have more than one object to return at the end of the function,
we can wrap these objects up together in another single object.

Consider 
\begin{verbatim}
GetTan <- function (x)
  {
    vT <- tan(x)

    vTT <- "your face is a tangent!"
    
    return(list(vT, vTT
                )
           )
  }

GetTan(1)
\end{verbatim}

We return the output as a list (because the objects had different
modes) and we use something like \verb=lOutput2 <- GetTan(4)= to save
it for future use.

If you pass a function an argument that it was not expecting based on the
function definition, you will get an error.

If you don't pass a function an argument that it was expecting based
on the function definition, you will get an error.

\subsection{Scope}
There are some very important rules to follow when using functions to
keep things clear. First, realize that a function is a temporary
workspace (or environment in \R{}-speak). The arguments are one-way
traffic onto this island. The \texttt{return} statement is one-way
traffic off the island.
\begin{enumerate}
\item any object not created by and contained within your function
  should be passed to it as an argument

\item any object(s) to be used later should be returned by your
  function

\item functions (with the exceptions of \texttt{save}, \texttt{plot},
  etc.) should not have side-effects
\end{enumerate}

Consider this code:
\begin{verbatim}
vThing <- 4

GetSquare <- function()
  {
    vThing ^ 2
  }

GetSquare()

rm(vThing)

GetSquare()
\end{verbatim}

We violated our rule of relying on other objects only through the
arguments. And, we can see how easily our function becomes worthless
if we don't listen to that rule.

Using the \verb!<<-! operator is a quick way to violate our other
rules.

Try:
\begin{verbatim}
BadFunc <- function()
  {
    cat("i woke up at 8am and all i got was this lousy R function")
    
    mLetters <<- "gone! mwahhahhah"
  }
\end{verbatim}

Look at \texttt{mLetters} from before. We changed it even though we
didn't pass the function any arguments, we didn't return any object as
output, and we didn't assign output to an identifier. Had we used
\verb=<-= instead of \verb=<<-=, we would only have altered the object
\texttt{mLetters} within the function environment so nothing would
have been overwritten.




%%% Local Variables: 
%%% mode: latex
%%% TeX-master: "../full_course"
%%% End: 

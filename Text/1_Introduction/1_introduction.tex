\part{An Introduction to Using \R}

This part of the short course introduces you to \R{} and provides a
basic overview of its use.

\section{What is \R?}

\begin{itemize}
\item a statistical scripting language (i.e.\ it was created with data
  in mind and you can use the language interactively)

\item a statistical environment (i.e.\ it is a program you open up and
  interact with)

\item open-source, free, compatible with most platforms, and
  constantly updated

\item the most common platform for quantitative work in political
  science (or at least I am claiming this)

\end{itemize}

You can find out more about \R{} at \url{http://www.r-project.org/}.

\section{Why Would You Use \R{}?}
\begin{itemize}
\item you will have to use it for PSC 405 and PSC 505 problem sets

\item it is the ``go to'' program for non-standard numerical analysis
  in political science

\item it was created with data in mind (so you aren't forcing a square
  peg in a round hole)

\item the language and your code is as much a way of communicating a
  set of instructions to other researchers as it is to your computer

\item the graphics capabilities of \R{} are un-rivaled among statistics
  software

\item \R{} can be anything you want it to be because it is so easily
  extended

\end{itemize}

Since you are now convinced that you would like to use \R{}, you can
download it at \url{http://www.r-project.org/index.html}. However, you
won't need to download it if you are in \tsl{}. It is
already installed on these computers.

\section{\R~and Your Workflow}

The role that \R{} will play in your work and how it will fit into your
workflow will vary. However, there is a general structure to it all.

\begin{enumerate}
\item come up with a numerical problem to solve
\item conceptualize a set of steps to solve this problem
\item translate these steps into the \R{} scripting language
\item run your \R{} code
\item debug your \R{} code
\item produce graphics (if applicable)
\item consume you results (either textual or graphical)
\end{enumerate}

\paragraph{Step 1.} \R{} will not tell you what you want to do. It can
help you get there, but you simply must know where you want to go. The
problem to be solved can range from \textit{calculating the sampling
  distribution of the sample mean of samples of five random draws from
  a distribution} to \textit{identifying the treatment effect of being exposed
to negative campaign advertisements on voter turnout}. In general, it
is a bad sign to have already opened \R{} if you don't know what you
want to do.

\paragraph{Step 2.} If \R{} is your first computer language, this step
will likely be tricky. This is all about translating a
human understanding of a problem into a machine-ready algorithm. For
some people this is easier than it is for others. For all people,
though, practice makes it easier.

\paragraph{Step 3.} This will be the hardest step. As you use \R{}, you
understanding of what is and is not possible in \R{} will evolve. The
code you write will take on incredibly different styles.

\par Because you are writing your \R{} code in a text-editor it will
be a plain text file (like in \LaTeX{}!).

\par When you write your \R{} code, I recommend doing several things:
\begin{itemize}
\item \textbf{do not use} the feature-less script editor in \R{}
  itself

\item \textbf{do use} a feature-full editor that integrates well
  enough with \R{} (e.g.\ RWinEDT on \tsl{} computers)

\item \textbf{do use} an editor to write your code and then run it at
  the \R{} prompt (this maintains transparency and reproducibility).

\end{itemize}

\paragraph{Step 4.} Running your code can happen at an interactive
\R{} shell or in ``batch mode''. The latter option will most likely
not be used by you any time soon. For our purposes we will run code
interactively. But, that is different from performing operations ``on
the fly''. If you write your code in RWinEdt, the toolbar icons
provide a simple way of running your already written code at the
interactive shell.

\paragraph{Step 5.} There is a built-in debugger in R and there are
some external packages which provide debugging facilities. I neither
use nor recommend these. For the most part, there is nothing you can't
debug with just some cleverness and the interactive command line. In
reality, Steps 4 and 5 are iterative and repetitive. In it's idealized
form, you might write your code in a single shot and run it without
errors. In practice, this will not happen. The point is, though, that
for any given project, you'll have it set up so that it can finally
run without intervention or correcting. It will be as if you wrote
completely correct code and sent it all to \R{}.

\paragraph{Step 6.} \R{} is known for the ability to produce great
graphical output. There are several different ``frameworks'' to
produce graphics. We will only cover the \textit{base} graphics in
this tutorial.

\paragraph{Step 7.} How you consume your output is up to you. But,
consumption is not a one-shot deal. Therefore, part of this step is
ensuring reproducibility. Also, you will want to ``export'' the
product of your hard work to a non-\R{} format. This can be a plain
text file of your results or a \texttt{.pdf} file with a figure you
created.

\section{The VERY Basics of the \R{} Interpreter}

\subsection{The Prompts}
\R{} tells you it is ready for input by displaying a
``\texttt{>}''. \R{} knows its own language, so if you hit
\textit{enter} and you expect something to evaluate, but \R{} keeps
giving you a ``\texttt{+}'' (to tell you it needs more) there is a
syntax error. To get out of the perpetual ``\texttt{+}'' prompts, hit
\textit{ctrl} + \textit{c} or \textit{esc}. When you get this problem,
there is likely to be a missing ``\verb="='', ``\verb=,='', or
``\verb=)=''.

At the end of a command, you hit \textit{enter} if you are at the
command line. If you are writing your script file in a text editor,
simply hit \textit{enter} to add a line break and start a new line.

\subsection{Comments}
\R{} interprets anything after a ``\texttt{\#}'' as a comment. The
comment ends and \R{} begins to interpret code once a new line is
started. Multi-line comments require a new comment character on each
line.

\subsection{Spaces}
\R{}, in general, ignores spaces between correctly entered objects and
operators. Spaces are not part of the names of objects or
operators. Spaces may be contained in character strings and they
matter.

\subsection{Multiple Commands on a Physical Line}
In \R{}, you can place two (or more) complete sets of commands on the
same physical line (i.e.\ without hitting \textit{enter}). You do this
by separating them with a ``\verb=;=''. However, this is not
recommended and the utility of it is questionable.


\section{Exercises}

These short exercises are intended to familiarize you with entering
commands at the prompt and the various kinds of windows that might
open as a result.

\begin{enumerate}
\item Start \R{} on the machines in \tsl{}.

\item If the \texttt{RWinEdt} package is not automatically loaded,
  type \texttt{library(RWinEdt)}. Hit \textit{enter}.

\item Type \texttt{sessionInfo()} at the command prompt (i.e.\
  \texttt{>}) and hit \textit{enter}. Your are running the function
  \texttt{sessionInfo} without any arguments. \R{} is ``computing''
  the output of the function and then ``printing'' the output with an
  ``appropriate'' method.

\item Type \texttt{citation()} at the command prompt and hit
  \textit{enter}. Recall the mention of BibTeX from the \LaTeX{}
  course? \R{} is ``computing'' the output of the function
  \texttt{citation} and then ``printing'' the output with an
  ``appropriate'' method.

\item Enter \texttt{demo(graphics)} at the prompt. Progress through
  the various demonstrated plots until you have seen them all. Close
  the plot window. \R{} is ``computing'' the output of the function
  \texttt{demo} which has been passed the argument
  \texttt{graphics}. In fact, \texttt{graphics} is a package which
  contains functions for creating graphics. Would you like to know
  something about it? Enter \texttt{?graphics} at the command
  line. This displays the help page for the \texttt{graphics}
  package. Close this.

\item Enter the following in WinEdt. You do not need to save this R
  script.

\begin{verbatim}
###
### Region 1
###
# is this 2 or 3?
1 + 1 # + 1
###

###
### Region 2
###
  1+ 1   +
1
###

###
### Region 3
###
"asdf " == "asdf"
###
\end{verbatim}

\item Evaluate each ``region'' one at a time. So, select the region
  with your cursor, and press the RWinEdt toolbar button with the word
  ``paste'' on it. This pastes the selected region into the \R{}
  shell.

\item Type \texttt{1 +} and hit \textit{enter}. Hit \textit{enter} a
  number of times. How do you get out of this?

\item Enter \texttt{q()} at the prompt. Respond ``\texttt{N}'' (no).
\end{enumerate}

%%% Local Variables: 
%%% mode: latex
%%% TeX-master: "../full_course"
%%% End: 


%%%%%%%%%%%%%%%%%%%%%%%%%%%%%%%%%%%%%%%%%%%%%%%%%%%%%%%%%%%%%%%%%%%%%%%%%%%%%%%% 
%%%%%%%%%%%%%%%%%%%%%%%%%%%%%%%%%%% PREAMBLE %%%%%%%%%%%%%%%%%%%%%%%%%%%%%%%%%%%
%%%%%%%%%%%%%%%%%%%%%%%%%%%%%%%%%%%%%%%%%%%%%%%%%%%%%%%%%%%%%%%%%%%%%%%%%%%%%%%% 


%%%%%%%%%%%%%%%%%%%%%%%%%%%%%%%%%%%%%%%%% 
%%%%%%%%%%%%%%%%% CLASS %%%%%%%%%%%%%%%%%
%%%%%%%%%%%%%%%%%%%%%%%%%%%%%%%%%%%%%%%%% 

\documentclass[10pt,letterpaper]{article}

%%%%%%%%%%%%%%%%%%%%%%%%%%%%%%%%%%%%%%%% 
%%%%%%%%%%%%%%% PACKAGES %%%%%%%%%%%%%%%
%%%%%%%%%%%%%%%%%%%%%%%%%%%%%%%%%%%%%%%% 

\usepackage[debugshow, final]{graphicx}
\usepackage{amsmath, amssymb, amsthm, amsfonts} 
\usepackage{enumerate}
\usepackage{fancybox} 
\usepackage{ltablex}
\usepackage{fancyhdr}
\usepackage{setspace} 
\usepackage[top=1in,left=1in,bottom=1in,right=1in]{geometry} 
\usepackage{natbib}
\usepackage{setspace} 
\usepackage{textcomp} 
\usepackage{pdflscape}
\usepackage{lscape}
\usepackage[table]{xcolor}
\usepackage[T1]{fontenc}
\usepackage{subfigure}
\usepackage{hyperref}
\usepackage[all]{hypcap}
\usepackage{listings}
% \usepackage{pxfonts}

%%%%%%%%%%%%%%%%%%%%%%%%%%%%%%%%%%%%%%%% 
%%%%%%%%%%%%%%% SETTINGS %%%%%%%%%%%%%%%
%%%%%%%%%%%%%%%%%%%%%%%%%%%%%%%%%%%%%%%% 

\pagestyle{fancy}
\lhead{}
\chead{\texttt{R} Course Problem Set (Answer Key)}
\rhead{\today}
\cfoot{\thepage}
\rfoot{Prepared by J.\ P.\ Olmsted}

\author{}
\title{}
\date{}

%%%%%%%%%%%%%%%%%%%%%%%%%%%%%%%%%%%%%%%%% 
%%%%%%%%%%%%% CUSTOM MACROS %%%%%%%%%%%%%
%%%%%%%%%%%%%%%%%%%%%%%%%%%%%%%%%%%%%%%%% 

\let\margin\marginpar
\newcommand\myMargin[1]{
  \margin{\singlespace\raggedright\scriptsize
    \textsf{#1}}
} 
\renewcommand{\marginpar}[1]{\myMargin{#1}}

\newcommand\blurb[1]{
  \begin{center}
    \parbox[h]{.75\linewidth}{
      \footnotesize \singlespace
      #1  
    }
  \end{center}
}

%%%%%%%%%%%%%%%%%%%%%%%%%%%%%%%%%%%%%%%%%%%%%%%%%%%%%%%%%%%%%%%%%%%%%%%%%%%%%%% 
%%%%%%%%%%%%%%%%%%%%%%%%%%%%%%%%%%% CONTENT %%%%%%%%%%%%%%%%%%%%%%%%%%%%%%%%%%%
%%%%%%%%%%%%%%%%%%%%%%%%%%%%%%%%%%%%%%%%%%%%%%%%%%%%%%%%%%%%%%%%%%%%%%%%%%%%%%% 

\begin{document}

%% \begin{titlepage}
%%   
%% \end{titlepage}

\section*{Instructions}
\paragraph{Submission.} All responses to these problems must be
received via email in my inbox by Friday, January 28th by 5pm. A
response should be in the form of a working, self-sufficient
\texttt{R} script. Because this is the only file that needs to be
turned in, you should place all information to identify whose
submission it is inside comments in the script.

\paragraph*{Resources.} The only external objects/data/functionality
that you are allowed to use are \texttt{R} packages from
\texttt{CRAN}. However, you do not need any at all.

\paragraph*{Solutions.} Textual solutions to the problems should be
placed as a comment after the code used to compute the
answer. Graphical solutions should be comprised of the code used to
generate the plots.

\paragraph*{Incentive Structure.} The questions will each be worth
different point values. Partial work can receive partial credit at my
discretion. Two people will receive Starbucks gift cards (5\$). The
decision rule used to draw the two winners will be as follows:
\begin{enumerate}
\item All valid problem sets will be graded and raw points assigned to
  each person.
\item Each point is worth one lottery ticket with your name on it.
\item The total number of points possible is irrelevant except that it
  places an upper limit on the number of lottery tickets you can
  receive.
\item The winner of the first gift card will be drawn from all of the
  lottery tickets.
\item The remainder of this winner's lottery tickets will be discarded.
\item Another winner will be drawn from the remaining tickets using
  the same procedure.
\end{enumerate}

Consequently, there is incentive to compete even if you know you won't
have time to work each problem out.

\paragraph{Assistance.} You \textbf{may not} ask any other first-year
student any questions regarding this problem set except ``are you done
yet?''  until after the deadline. You \textbf{may} ask other students
in the program for their help and thoughts, but you must disclose the
nature of this assignment! You \textbf{should} ask me questions since
I have solutions already prepared!

\paragraph{Solutions.} Where possible, the provided solutions reflect
an approach to solving the problem which is like the material we
covered in class. This does not mean there are not more concise or
more efficient ways of solving the problem. For example, loops are
employed here more than they should be in practice.

\section*{Example}

The following is an example question and the example code that might
correspond to a full credit response.

\subsection*{Example Question}
Consider a sequence such that the $i^{th}$ term, $a_i$, is given
by
\[
a_i = log\left((i^{i - 1})^{1/2}\right).
\]

Find the sum of the first 50 terms, that is, find
\[
\sum_{i = 1}^{50} a_i.
\]
\subsection*{Example Response}
\lstset{ %
  language=R,
  basicstyle=\scriptsize\ttfamily, % the size of the fonts that are used
  commentstyle=\ttfamily\color{gray},
  % for the code
  numbers=left,                   % where to put the line-numbers
  numberstyle=\ttfamily\color{gray}\footnotesize,      % the size of the fonts that are used
  % for the line-numbers
  stepnumber=1,                   % the step between two
  % line-numbers. If it's 1 each line
  % will be numbered
  numbersep=5pt,                  % how far the line-numbers are from
  % the code
  backgroundcolor=\color{white},  % choose the background color. You
  % must add \usepackage{color}
  showspaces=false,               % show spaces adding particular
  % underscores
  showstringspaces=false,         % underline spaces within string
  showtabs=false,                 % show tabs within strings adding
  % particular underscores
  frame=single,                % adds a frame around the code
  tabsize=2,                % sets default tabsize to 2 spaces
  captionpos=b,                   % sets the caption-position to bottom
  breaklines=true,                % sets automatic line breaking
  breakatwhitespace=false,        % sets if automatic breaks should only
  % happen at whitespace
  title=\lstname,                 % show the filename of files included
  % with \lstinputlisting;
  % also try caption instead of title
  escapeinside={},         % if you want to add a comment within
  % your code
  keywordstyle={},
  morekeywords={*,...}            % if you want to add more keywords to
  % the set
}

\lstinputlisting{./hw_example.R}

\section*{Problems}

\subsection*{Problem 1}
Consider random samples of size 10 from the F distribution with 3 and
7 degrees of freedom. Use 1,000 such samples to calculate the standard
deviation of the sampling distribution of the sample mean. (\textbf{10
  points})

% \lstinputlisting{./hw1.R}

\subsection*{Problem 2}
Consider the 26 letters of the Modern English alphabet as defined here
(\url{http://en.wikipedia.org/wiki/English_alphabet}). Assume the
following definitions:
\begin{description}
\item [Letter Value] The integer value of the index corresponding to a
  particular letter given the standard \textit{a, b, c, d, \ldots}
  order of the letters.\footnote{So, $\textrm{LV}(a) = 1$ and
    $\textrm{LV}(z) = 26$ where $\textrm{LV}$ is the letter value
    function which maps a letter to its letter value.}

\item [Text Sum] The sum of the sequence of letter values
  corresponding to a sequence of letters comprising a portion of text
  where all whitespace and punctuation is ignored.
\end{description}

Find the \textit{Text Sum} corresponding to the main body of text on
\url{http://www.rochester.edu/college/gradstudents/jolmsted/research/}
which begins ``My primary academic interests \ldots'' and ends
``\ldots original spatial voting model.'' (\textbf{20 points})

\paragraph{Hint.} This problem requires ``string manipulation''.

% \lstinputlisting{./hw2.R}

\subsection*{Problem 3}

Construct a function to compute whether a given integer input is
``odd'' or ``even''. Use this function to confirm that the sum of the
odd numbers between the 1 and 20, inclusive, is 100. (\textbf{5
  points})

% \lstinputlisting{./hw3.R}

\subsection*{Problem 4}
Consider $x = (x_1, x2_, \ldots, x_{1,000,000})$ and $y = (y_1, y_2,
\ldots, y_{1,000,000})$ where
\[
x_i \sim N(0, \sigma = 1),
\]
and
\[
y_i \sim N(x_i, \sigma = s).
\]
Find some value of $s$ such that $\rho(x, y) \in (0.3, 0.4)$, where
$\rho$ is the sample correlation between the vectors. You may consider
$s$ a solution if three runs of the data-generating process produce
$\rho$'s that satisfy the constraint. (\textbf{20 points})

\paragraph{Note.} Technically, $\rho$ is a random variable because $x$
and $y$ are random variables and it has non-zero density over the unit
interval. However, for an appropriate value of $s$, the use of
1,000,000 draws ensures that we can constrain the sampling
distribution of $\rho$ to be sufficiently within the desired interval
that we can ignore the chance that we will draw a sample which results
in a correlation in the interval sometimes and not in others. That is,
if you run the data generating process for a fixed $s$ several times
and you get a $\rho$ that satisfies this constraint you can count this
as an answer.

% \lstinputlisting{./hw4.R}

\subsection*{Problem 5}
Base \texttt{R} and \texttt{R} packages provide datasets that don't
need to be loaded from external files. They can be accessed with the
function \texttt{data()}. Load the \texttt{airmiles} dataset and the
\texttt{discoveries} dataset. The ``data'' that these two provide are
two appropriately named vectors that are also \texttt{timeseries}
objects. Use \verb=ts.intersect()= to merge the time series and then
coerce this object to a dataframe object.

\begin{enumerate}
\item Code a new variable in the dataframe object such that it takes
  on a value of 1 if both the number of \texttt{airmiles} is odd and
  the number of \texttt{discoveries} is odd, the value 2 if both of
  the other variables of interest are even, and the value 3 if the
  variables of interest are mixed. Call this variable
  ``yeartype''. (\textbf{5 points})

\item If \texttt{yeartype} were a factor variable, what would the
  summary statistics for that variable be? (\textbf{5 points})

\item If \texttt{yeartype} were a numeric variable, what would the
  summary statistics for that variable be? (\textbf{5 points})

\item Calculate summary statistics conditional on the value of
  yeartype. (\textbf{5 points})

\item Plot a graph of \texttt{discoveries} vs.\ \texttt{airmiles} with
  the following characteristics:
  \begin{itemize}
  \item the axes and title are given informative names (\textbf{5 points})
  \item each $(x,y)$ point has a different character based on its \texttt{yeartype} (\textbf{5 points})
  \item each $(x,y)$ point has a different color based on whether the
    corresponding \texttt{airmiles} value is greater than the overall
    average over each year or less than the overall average. Equality
    can be assigned to either group. (\textbf{5 points})
  \end{itemize}
\end{enumerate}

% \lstinputlisting{./hw5.R}

\section*{Problem 6}
Consider a sequence $b = (b_1, b_2, \ldots, b_k)$ where
\[
b_i = i + k.
\]
Find the $k$ that maximizes the sum of the sequence subject to the
ceiling $L$: $\sum_{i = 1}^{k} b_i < L$. This is a strict inequality!
Find these $k$'s for $L \in \lbrace 10, 100, 1000 \rbrace$. (\textbf{15 points})

%\lstinputlisting{./hw6.R}


%% \bibliographystyle{}
%% \bibliography{}

\end{document}

%%%%%%%%%%%%%%%%%%%%%%%%%%%%%%%%%%%%%%%%%%%%%%%%%%%%%%%%%%%%%%%%%%%%%%%%%%%%%%% 
%%%%%%%%%%%%%%%%%%%%%%%%%%%%%%%%%%%%%%%%%%%%%%%%%%%%%%%%%%%%%%%%%%%%%%%%%%%%%%% 
%%%%%%%%%%%%%%%%%%%%%%%%%%%%%%%%%%%%%%%%%%%%%%%%%%%%%%%%%%%%%%%%%%%%%%%%%%%%%%% 

